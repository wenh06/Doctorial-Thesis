\chapter{超曲面上的重数估计}
\label{chapter:main result}
为了研究射影超曲面上的重数计数问题,我们先介绍一些关于超曲面上点的重数的一些事实。

\section{超曲面截影上的重数}
设$k$为任意的一个域,$f\in k[T_0,\cdots,T_n]$为一个次数等于$\delta$的非零齐次多项式。考虑概型
\begin{equation}
X = V(f) = \proj\left(k[T_0,\cdots,T_n]/(f)\right).
\end{equation}
事实上,概型$V(f)$是$n$维射影空间$\mathbb P^n_k$的维数纯为$n-1$的闭子概型,从而是射影空间$\mathbb{P}^n_k$内的一个超曲面,被称为为由齐次多项式$f$定义的射影超曲面。可以证明,射影超曲面$X = V(f)$的次数等于定义它的齐次多项式$f$的次数$\delta$(详见参考文献~\inlinecite{GTM52}的第一章的 Proposition 7.6)。

令$\alpha\in[0,\delta]\cap \mathbb N$为一个非负整数。我们记$\mathcal{T}^{\alpha}(f)$为形如
\begin{equation}
\partial^{|I|}_{I} f := \dfrac{\partial^{|I|}f}{\partial T^I} = \dfrac{\partial^{i_0+\cdots+i_n}f}{\partial T_0^{i_0}\cdots\partial T_n^{i_n}}
\end{equation}
的齐次多项式$f$的阶数为$\alpha$的偏导数张成的$k$-线性空间,其中$I = (i_0,\cdots,i_n)\in\mathbb{N}^{n+1}$且满足$|I| = i_0+\cdots+i_n = \alpha$。线性空间$\mathcal{T}^{\alpha}(f)$中的元素都是次数等于$\delta-\alpha$的齐次多项式。

下面,我们要引入一个决定超曲面上点的重数的显式的准则。

\begin{proposition}[\inlinecite{Liu-multiplicity}, Corollaire 5.4] \label{taylor expansion}
设域$k$的特征为$0$,$X\hookrightarrow\mathbb P^n_k$为由次数等于$\delta$的齐次多项式$f$定义的超曲面,$\xi\in X$为超曲面上的一个点,$\alpha$为区间$[0,\mu_{\xi}(X)-1]$内的非负正整数。那么任取线性空间$\mathcal{T}^{\alpha}(f)$中任意一个非零元素$g$(一个次数等于$\delta-\alpha$的齐次多项式),点$\xi$总是包含在由$g$定义的超曲面$X_{\alpha,g}$上。而当$\alpha = \mu_{\xi}(X)$时,情况正好相反,总存在一个非零元素$g'\in\mathcal{T}^{\mu_{\xi}(X)}(f)$,使得点$\xi$不落在由$g'$定义的超曲面上。
\end{proposition}

\begin{remark}
特别地,如果$\mu_{\xi}(X) > 1$,也就是说点$\xi$是一个奇异点,那么对$\mathcal{T}^1(f)$的每个元素$g$来说,点$\xi$总是落在由$g$定义的超曲面上。
\end{remark}

\begin{remark}
上述命题一个直接的结果就是,对于命题中出现的超曲面$X_{\alpha,g}$,总有
\begin{equation}
\mu_{\xi}(X_{\alpha,g}X') \geqslant \mu_{\xi}(X) - \alpha, \quad \forall \alpha \in [0,\mu_\xi(X)-1].
\end{equation}
\end{remark}

\section{相交树的构造}
\label{construction of intersection trees}
由于有命题\ref{taylor expansion},我们可以构造一族相交树,或者说一个森林,用来解决我们所考虑的重数计数问题。首先,我们需要把这族相交树的根都确定下来。为了达到这个目的,我们需要以下的命题

\begin{proposition}[\inlinecite{Liu-multiplicity}, Lemme 5.8] \label{construction of roots}
设$K$为一个数域,$f\in K[T_0,\cdots,T_n]$为一个次数等于$\delta$的齐次多项式。记$V(f)$为由齐次多项式$f$定义的超曲面。如果超曲面$V(f)$的奇点集维数等于$s$,那么存在$f$的一组方向导数$g_1,\cdots, g_{n-s-1}\in\mathcal{T}^1(f)$,使得
\begin{equation}
\dim(V(f)\cap V(g_1)\cap\cdots\cap V(g_{n-s-1}))=s.
\end{equation}
换而言之,$V(f)\cap V(g_1)\cap\cdots\cap V(g_{n-s-1})$是一个完全交。
\end{proposition}

设$K,f$如以上命题中所设,我们将在以后的论述中由齐次多项式$f$定义的超曲面$V(f)$记作$X$,即
\begin{equation}
X = \proj\left(K[T_0,\cdots,T_n]/(f)\right).
\end{equation}
我们将把$X$的正则点集记作$X^{\mathrm{reg}}$,把$X$的奇点集记作$X^{\mathrm{sing}}$。此外,为了符号统一便于书写,我们还把命题\ref{construction of roots}中出现的超曲面$V(g_i)$写作$X_i$,其中$i=1,\cdots,n-s-1$。由Jacobian判别准则(详见参考文献~\inlinecite{LiuQing}的 Theorem 4.2.19),有
\begin{equation}
X^{\mathrm{sing}} \subseteq X\cap X_1\cap\cdots\cap X_{n-s-1}.
\end{equation}

我们接下来就以集合$\mathcal C(X\cdot X_{1}\intersect X_{n-s-1})$中的所有元素,也就是射影空间$\mathbb{P}_K^n$中的相交积$X\cdot X_{1}\intersect X_{n-s-1}$的所有不可约分支,为根节点,来构造森林$\mathscr{F}_{\mathcal C(X\cdot X_{1}\intersect X_{n-s-1})} = \{\mathscr T_C\}_{C\in\mathcal C(X\cdot X_{1}\intersect X_{n-s-1})}$。其中,森林中每棵(相交)树$\mathscr T_C$的下标表示它以$C\in\mathcal C(X\cdot X_{1}\intersect X_{n-s-1})$为根节点。

我们归纳地来构造相交树里深度大于等于$1$的顶点。注意到所有深度等于$0$的点,即根节点,我们都已经取好了。假设$M$是我们已经在某一个相交树$\mathscr T_C$上构造好了的一个顶点。

对于$X$的一个整的子概形$M$,我们用$M^{(a)}$来表示$M$中那些在$X$中重数等于$M$在$X$中重数的点的集合,用$M^{(b)}$来表示$M$中那些在$X$中重数大于等于等于$\mu_M(X)+1$的点的集合。用数学符号来表示,就是
\begin{gather}
M^{(a)} = \{ \xi\in M \ |\ \mu_{\xi}(X) = \mu_{M}(X) \} ,\\
M^{(b)} = \{ \xi\in M \ |\ \mu_{\xi}(X) \geqslant \mu_{M}(X)+1 \}.
\end{gather}
此外,设$L/K$为数域的扩张,我们分别记$M^{(a)}(L)$,$M^{(b)}(L)$为$M^{(a)}$,$M^{(b)}$的$L$-有理点的集合。那么有
\begin{equation}
M(L) = M^{(a)}(L)\bigsqcup M^{(b)}(L).
\end{equation}

由命题\ref{taylor expansion},容易推出$M^{(a)}$在$M$中是稠密的,因为$M^\mathrm{reg}$是在$M$中稠密的,并且$M^\mathrm{reg}$中所有点在$X$中的重数都是$\mu_M(X)$。而$M^{(b)}$的维数则要小于等于$\dim (M)-1$。

回到我们的构造。对于顶点$M$,我们把他视作$X$的整闭子概型。考虑集合$M(\overline K)$。如果它的子集$M^{(b)}(\overline K)=\emptyset$,那么我们令$M$为叶节点,它的标签取为空集。也就是说$M$没有子节点,$M$所在的相交树这一分支构造完毕。如果$M^{(b)}(\overline K)\neq\emptyset$,那么我们任取其中的一点$\xi \in M^{(b)}(\overline K)$。由集合$M^{(b)}$的定义,我们有$\mu_M(X) < \mu_{\xi}(X) \leqslant \delta$。根据命题\ref{taylor expansion},对集合$M^{(a)}(\overline K)$中取定的一点$\xi'\in M^{(a)}(\overline K)$,我们可以在齐次多项式阶数为$\delta-\mu_M(X)$的偏导数张成的$K$-线性空间中找到某个元素$h \in \mathcal{T}^{\delta-\mu_M(X)}(f)$,使得由齐次多项式$h$定义的超曲面$V(h)$不包含点$\xi'$。那么同时,$V(h)$不包含$M$的广点。通过比较$M$与$V(h)$的维数可以知道,$M$与$V(h)$是正常相交的。线性空间$\mathcal{T}^{\delta-\mu_M(X)}(f)$中的非零元都是$\mu_M(X)$次齐次多项式,因此$\deg(h) = \mu_M(X) \leqslant \delta-1$。我们将超曲面$V(h)$取作顶点$M$的标签,记为$\widetilde{M}$。我们把顶点$M$的子节点取作相交积$M\cdot\widetilde{M}$的不可约分支。顶点$M$与其子节点相连的边的权重被定义作相应的相交重数。

值得注意的是,在构造相交树的过程中,每个顶点的标签如果不是空集,那么就是一个超曲面,维数等于$n-1$,因此集合$\mathcal{C}_t$中所有元素,即所有深度等于$t$的顶点,都是$s-t$维的,其中$1\leqslant t\leqslant s$为一个正整数。于是,以上的构造会在有限步内终结。于是,依照上文所叙述的程序,我们就可以把这族相交树$\{\mathscr T_C\}_{C\in\mathcal C(X\cdot X_{1}\intersect X_{n-s-1})}$都构造出来了。需要特别指出来的是,以上对于森林$\mathscr{F}_{\mathcal C(X\cdot X_{1}\intersect X_{n-s-1})} = \{\mathscr T_C\}_{C\in\mathcal C(X\cdot X_{1}\intersect X_{n-s-1})}$的构造并不是唯一的,因为顶点标签的选取并不唯一。

\begin{remark} \label{philosophy of intersection forest}
如此构造相交树的思路可以简述如下:

为了方便,我们来考虑稍微简单一些的$K = \mathbb{Q}$时的有理点重数的计数。首先,在最简单的情况,即所有的$M \in \mathcal{C}(X\cdot X_1\intersect X_{n-s-1})$都满足$M^{(b)} = \emptyset$,那么直接有(下式中为了方便,记$X = X_0$)
\begin{align}
& \sum\limits_{\xi \in S(X;B)} \mu_{\xi}(X)(\mu_{\xi}(X)-1)^{n-s-1} \leqslant \sum\limits_M N(M;B) \prod\limits_i \mu_{M}(X_i) \\
& \ll_{n} B^{s+1} \sum\limits_M \deg(M) \cdot i(M; X_0\intersect X_{n-s-1}; \mathbb{P}^n_{\mathbb{Q}}) \\
& \ll_{n} B^{s+1} \prod\limits_i \deg(X_i) \ll_{n} \delta(\delta-1)^{n-s-1} B^{s+1} \\
& \ll_{n,K} \delta^{n-s} \max\{B, \delta-1\}^{s+1}.
\end{align}
这直接是我们想要的(有理点重数计数的)结论。

如果情况不是那么好,存在某些相交积的不可约分支$M \in \mathcal{C}(X\cdot X_1\intersect X_{n-s-1})$使得$M^{(b)}(K) \neq \emptyset$,即$M$里面有一些点,它们的重数不能被$M$的广点的重数所控制,那么上面的不等式便不成立了。上面不等式第一行的左边求和号下的$\xi$加上额外的限制条件:$\mu_{\xi}(X) = \mu_M(X)$,对$\xi \in M$,才能使上面的不等式成立。这样一来,我们只得到一个部分的估计,还剩下那些没有被估计到的点,则需要想别的办法对其重数进行控制。而这个办法,就是我们在前文构造相交树的时候采用的方法:找一个与$M$正常相交的超曲面,而且他们的交正好把这些重数不能被$M$的广点的重数所控制的点包住。然后对相交积的每个不可约分支,再次尝试用广点的重数对其他点的重数进行控制。如此重复进行下去,直到过程终止。而我们在相交树构造的过程中已经提到了,这个过程一定会在有限步内终止。
\end{remark}

\begin{remark} \label{simple facts on forest}
关于上文构造的森林$\mathscr{F}_{\mathcal C(X\cdot X_{1}\intersect X_{n-s-1})} = \{\mathscr T_C\}_{C\in\mathcal C(X\cdot X_{1}\intersect X_{n-s-1})}$,有两个需要注意到的,以后会反复使用的性质,在这里强调并列举如下:
\begin{itemize}
\item 森林中顶点的每个非空的标签对应的闭子概型(实际上也是$\mathbb{P}_K^n$中的超曲面)的次数都小于等于$\delta-1$;
\item 每棵相交树中深度等于$t$的顶点(也就是说集合$\mathcal{C}_t$中的元素)对应的闭子概型的维数都等于$s-t$。
\end{itemize}
\end{remark}

下面我们还要给一个关于集合$\mathcal Z_*$性质的关键性的引理(集合$\mathcal Z_*$的定义见前文的定义\ref{def of Z_s})。集合$\mathcal Z_*$的这个性质在接下来将要证明的我们的主定理\ref{main theorem},进行相关的估计的时候,有基础性的作用。

\begin{lemma}[\inlinecite{Liu-multiplicity}, Lemme 5.9] \label{control of singular locus}
任取概型$X$的奇异点$\xi\in X^{\mathrm{sing}}(\overline{K})$,总是至少存在集合$\mathcal Z_*$中的一个顶点$Z$,使得$\xi\in Z^{(a)}(\overline K)$。
\end{lemma}
这是一个比较关键的引理,我们下面来证明它。
\begin{proof}
我们有三个观察:
\begin{itemize}
\item 首先,正如我们在上文已经说过,由Jacobian判别准则,有
\begin{equation}
X^\mathrm{sing}\subseteq X\cap X_1\cap\cdots\cap X_{n-s-1},
\end{equation}
因此,任取射影超曲面$X$的奇异点$\xi\in X^{\sing}(\overline{K})$,总存在$Z\in\mathcal C(X\cdot X_{1}\intersect X_{n-s-1})$,使得$\xi \in Z(\overline{K})$。
\item 其次,注意到集合$\mathcal{C}_t$中顶点所对应的闭子概型的维数都等于$s-t$(见注释\ref{simple facts on forest}),而且$s\leqslant n-2$,如果集合$\mathcal{C}_{n-2}$不等于空集,里面元素(相交树的顶点)所对应的闭子概型都是$0$维的,由有限个孤立点组成。这些闭点作为$X$的既约子概形,必然是正则的。也就是说在这种情况下,任取$Z \in \mathcal{C}_{n-2}$,有$Z^{(b)} = \emptyset$。
\item 假设$t$满足$\mathcal{C}_t = \emptyset$,但$\mathcal{C}_{t-1} \neq \emptyset$,那么由定义必然有
\begin{equation}
\forall Z \in \mathcal{C}_{t-1}, ~~ Z^{(b)} = \emptyset.
\end{equation}
\end{itemize}
于是,由以上三个观察我们能得出的结论是:对任意的奇异点$\xi\in X^{\sing}(\overline{K})$,总存在某个$w$,以及$Z\in \mathcal{C}_w$使得$\xi\in Z^{(a)}(\overline{K})$。

对于一个取定的奇异点$\xi$,令$w$为满足存在$Z\in \mathcal{C}_w$使得$\xi\in Z^{(a)}(\overline{K})$的最小的整数。如果$Z$可以落在集合$\mathcal{Z}_w \subseteq \mathcal{C}_w$中,那么我们就证明完毕了。

如若不然,那么对任意满足$\xi\in Z^{(a)}(\overline{K})$的$Z\in \mathcal{C}_w$,都有$Z\not\in \mathcal{Z}_w$。那么存在满足下面两个条件的极大的$w'$:
\begin{itemize}
\item $w' < w$;
\item 存在$Z' \in \mathcal{C}_{w'}$,使得$Z\subsetneqq Z'$,而且作为森林$\mathscr{F}_{\mathcal C(X\cdot X_{1}\intersect X_{n-s-1})}$中的顶点,$Z$不是$Z'$的子孙。
\end{itemize}
如果$\xi\in Z'^{(a)}(\overline{K})$,那么这会与$w$的极小性相矛盾。如果$\xi\in Z'^{(b)}(\overline{K})$,那么会有
\begin{equation}
\mu_{\eta_Z}(X) = \mu_Z(X) = \mu_{\xi}(X) > \mu_{Z'}(X) = \mu_{\eta_{Z'}}(X),
\end{equation}
其中$\eta_Z$与$\eta_{Z'}$分别为$Z$与$Z'$的广点。根据上文有关相交树的构造,特别是关于其构造思路的注释\ref{philosophy of intersection forest},我们知道$Z$一定会是$Z'$的子孙,这与$w'$的取法矛盾。证明完毕。
\end{proof}

\begin{remark}
引理\ref{control of singular locus}在刘春晖文章~\inlinecite{Liu-multiplicity}中是在有限域上证明的。事实上,这个结论在数域上仍然是成立的,因为文章~\inlinecite{Liu-multiplicity}中对这个引理的证明用到基域$k$的性质的地方,仅仅在于要求它是一个完全域。
\end{remark}

\section{重数的计数}
\label{counting multiplicity}
我们沿用本章的常用的记号。以下便是我们要证明的主定理的叙述

\begin{theorem} \label{main theorem}
令$K$为一个数域,$n \geqslant 2$,$\delta \geqslant 1$以及$s \geqslant 0$为整数。那么以下的不等式
\begin{equation} \label{estimate in main theorem}
\sum_{\xi\in S(X;D,B)} \mu_{\xi}(X)(\mu_{\xi}(X)-1)^{n-s-1} \leqslant \sum_{t=0}^s\max_{Z\in\mathcal Z_t}\left\{\frac{N(Z;D,B)}{\deg(Z)}\right\}\delta(\delta-1)^{n-s+t-1}.
\end{equation}
% \begin{eqnarray*}& &\sum_{\xi\in S(X;D,B)} \mu_{\xi}(X)(\mu_{\xi}(X)-1)^{n-s-1}\\
% &\leqslant&\sum_{t=0}^s\max_{Z\in\mathcal Z_t}\left\{\frac{N(Z;D,B)}{\deg(Z)}\right\}\delta(\delta-1)^{n-s+t-1}.\end{eqnarray*}
对于$P^n_K$中所有次数等于$\delta$,且奇点集维数等于$s$的既约的射影超曲面$X$都是成立的。上式中,符号$S(X;D,B)$与$N(X;D,B)$的定义分别在在式\eqref{S(X;D,B)}与式\eqref{N(X;D,B)}中给出,集合$\mathcal{Z}_t$的定义在定义\ref{def of Z_s}中给出。特别地,如果某个$\mathcal{Z}_t = \emptyset$,$0 \leqslant t \leqslant s$,那么我们约定$\max\limits_{Z\in\mathcal Z_t} \left\{ \frac{N(Z;D,B)}{\deg(Z)} \right\} = 0$。
\end{theorem}

\begin{proof}
对于$P^n_K$中任意的一个次数等于$\delta$,奇点集维数等于$s$的既约的射影超曲面$X$,我们假设已经由\S \ref{construction of intersection trees}中所描述的程序构造好了森林$\mathscr{F}_{\mathcal C(X\cdot X_{1}\intersect X_{n-s-1})} = \{\mathscr T_C\}_{C\in\mathcal C(X\cdot X_{1}\intersect X_{n-s-1})}$,使得森林中每棵相交树的根都来自集合$\mathcal C(X\cdot X_{1}\intersect X_{n-s-1})$。

首先,对于我们想要证明的不等式的左边,由于超曲面$X$中正则点的重数都等于$1$,于是我们有
\begin{equation} \label{first step}
\sum\limits_{\xi\in S(X;D,B)} \mu_\xi(X)(\mu_\xi(X)-1)^{n-s-1} = \sum\limits_{\xi\in S(X^\mathrm{sing};D,B)} \mu_\xi(X)(\mu_\xi(X)-1)^{n-s-1}.
\end{equation}

由引理\ref{control of singular locus},任取超曲面$X$上的一个奇异点$\xi\in X^{\mathrm{sing}}(\overline K)$,我们总能找到一个顶点,或者说一个闭子概型,$Z\in\mathcal{Z}_*$,使得$\xi\in Z^{(a)}(\overline K)$,因此我们又有
% \begin{equation} \label{final 1}
% \sum\limits_{\xi\in S(X^\mathrm{sing};D,B)}\mu_\xi(X)(\mu_\xi(X)-1)^{n-s-1} \leqslant \sum_{t=0}^{s} \sum_{Z\in\mathcal Z_t} \sum_{\xi\in S(Z^{(a)};D,B)} \mu_\xi(X)(\mu_\xi(X)-1)^{n-s-1}.
% \end{equation}
\begin{multline} \label{final 1}
\sum\limits_{\xi\in S(X^\mathrm{sing};D,B)} \mu_\xi(X)(\mu_\xi(X)-1)^{n-s-1}\\
\leqslant \sum_{t=0}^{s} \sum_{Z\in\mathcal Z_t} \sum_{\xi\in S(Z^{(a)};D,B)} \mu_\xi(X)(\mu_\xi(X)-1)^{n-s-1}.
\end{multline}

又由命题\ref{taylor expansion},对于每个$Z\in\mathcal Z_*$,我们有不等式
\begin{equation}
\mu_Z(X)-1\leqslant\mu_Z(X_i),
\end{equation}
对$i=1,\cdots,n-s-1$都成立。于是我们有不等式
\begin{equation} \label{final_1'}
\mu_Z(X)(\mu_Z(X)-1)^{n-s-1} \leqslant \mu_Z(X)\mu_Z(X_{1})\cdots\mu_Z(X_{n-s-1}).
\end{equation}

根据命题\ref{grassmanne}以及上面的不等式\eqref{final_1'},对于每个$t=0,\cdots,s$,我们有
\begin{align} \label{final 2}
& \sum_{Z\in\mathcal Z_t} \mu_Z(X)(\mu_Z(X)-1)^{n-s-1}\deg(Z) \nonumber \\
& \leqslant \sum_{Z\in\mathcal Z_t} \mu_Z(X)\mu_Z(X_{1})\cdots\mu_Z( X_{n-s-1})\deg(Z) \\
& \leqslant \deg(X)\prod_{i=1}^{n-s-1}\deg(X_i)\prod_{j=0}^{t-1}\max_{\widetilde{Z}\in \mathcal C'_t}\{\deg(\widetilde{Z})\} \\
& \leqslant \delta(\delta-1)^{n-s+t-1}.
\end{align}
这是因为集合$\mathcal C'_*$中所有的元素(相交树上顶点的标签,也是闭子概型)的次数都是小于等于$\delta-1$的。

把不等式\eqref{final 1}和不等式\eqref{final 2}结合起来看,便有
\begin{align} \label{final step}
& \sum_{t=0}^{s} \sum_{Z\in\mathcal{Z}_t} \sum_{\xi\in S(Z^{(a)};D,B)} \mu_\xi(X)(\mu_\xi(X)-1)^{n-s-1} \nonumber \\
= & \sum_{t=0}^{s} \sum_{Z\in\mathcal{Z}_t} \mu_Z(X)(\mu_Z(X)-1)^{n-s-1}N(Z^{(a)};D,B) \\
\leqslant & \sum_{t=0}^{s} \sum_{Z\in\mathcal{Z}_t} \mu_Z(X)(\mu_Z(X)-1)^{n-s-1}N(Z;D,B) \\
\leqslant & \sum_{t=0}^{s} \max_{Z\in\mathcal{Z}_t} \left\{\frac{N(Z;D,B)}{\deg(Z)}\right\} \left(\sum_{Z\in\mathcal{Z}_t} \mu_Z(X)(\mu_Z(X)-1)^{n-s-1}\deg(Z)\right) \\
\leqslant & \sum_{t=0}^s \max_{Z\in\mathcal{Z}_t} \left\{\frac{N(Z;D,B)}{\deg(Z)}\right\} \delta(\delta-1)^{n-s+t-1}.
\end{align}
把不等式\eqref{first step},\eqref{final 1}以及\eqref{final step}综合起来,就得到了我们想要证明的不等式。定理证毕。
\end{proof}

\section{有理点重数的计数}
在这一小节中,我们考虑定理\ref{main theorem}的一种相对简单一些的特殊情况,即有理点重数的计数问题。结合之前我们得到的关于一般化的 Schanuel 估计的一个一致性的结果,我们将要得到一个比原定理稍微精细一些的结果。自然,这个结果也是一个一致性的估计。

\begin{corollary} \label{main corollary}
设$n \geqslant 2$,$\delta \geqslant 1$以及$s \geqslant 0$为整数。那么以下的估计
\begin{equation} \label{estimate in main corollary}
\sum_{\xi\in S(X;B)} \mu_\xi(X)(\mu_\xi(X)-1)^{n-s-1} \ll_{n} \delta^{n-s}\max\{B,\delta-1\}^{s+1}.
\end{equation}
对于$\mathbb{P}^n_{\mathbb{Q}}$中所有次数等于$\delta$,且奇点集维数等于$s$的既约的射影超曲面$X$都是成立的。
\end{corollary}

\begin{proof}
% 由\S\ref{construction of intersection trees}中相交树的构造,集合$\mathcal{Z}_t, t=0,\cdots,s$,的每个元素$Z$都是一个局部完全交。再结合注释\ref{degree 1 estimate}中的论述,我们有如下的估计
对于$\mathbb{P}^n_{\mathbb{Q}}$中任何一个次数等于$\delta$,且奇点集维数等于$s$的既约的射影超曲面$X$,假设已经由\S \ref{construction of intersection trees}中所描述的程序构造好了森林$\mathscr{F}_{\mathcal C(X\cdot X_{1}\intersect X_{n-s-1})} = \{\mathscr T_C\}_{C\in\mathcal C(X\cdot X_{1}\intersect X_{n-s-1})}$,使得森林中每棵相交树的根都来自集合$\mathcal C(X\cdot X_{1}\intersect X_{n-s-1})$。考虑我们在定义\ref{def of Z_s}中特别取出来的森林中的顶点的子集$\mathcal{Z}_t$,对于每一个$Z \in \mathcal{Z}_t$,$t = 0,\ldots,s$,由定理\ref{refined Schanuel estimate for rational points}我们有估计$N(Z;B) = \#S(Z;B) \ll_{n} \deg(Z) B^{\dim(Z)+1}$,亦即
\begin{equation}
\frac{N(Z;B)}{\deg(Z)} \ll_{n} B^{\dim(Z)+1}, \quad B \geqslant 1.
\end{equation}
把这个估计代入上一小节证明的主定理\ref{main theorem}中,结合集合$\mathcal{C}_t$中(因此,集合$\mathcal{Z}_t$中)的每个顶点$Z$的维数$\dim(Z) = s - t$以及$s \leqslant n-2$的事实,我们有
\begin{eqnarray}
& & \sum_{\xi\in S(X;B)} \mu_\xi(X)(\mu_\xi(X)-1)^{n-s-1} \nonumber \\
& \leqslant & \sum_{t=0}^s \max_{Z\in\mathcal Z_t}\left\{\frac{N(Z;B)}{\deg(Z)}\right\}\delta(\delta-1)^{n-s+t-1} \\
& \ll_{n} & \sum_{t=0}^s B^{s-t+1}\delta(\delta-1)^{n-s+t-1} \\
& \leqslant & \begin{cases}
\delta(\delta-1)^n \sum\limits_{t=1}^{s+1} \left(\dfrac{B}{\delta-1}\right)^t & \text{若$B > \delta - 1$} \\
\delta(\delta-1)^{n-s-1}B^{s+1} \sum\limits_{t=0}^{s} \left(\dfrac{\delta-1}{B}\right)^t & \text{若$B < \delta - 1$} \\
\delta(\delta-1)^n(s+1) & \text{若$B = \delta - 1$}
\end{cases} \\
& \leqslant & \delta(\delta-1)^{n-s-1} \max\{B,\delta-1\}^{s+1} (s+1) \\
& \leqslant & (n-1) \delta^{n-s} \max\{B,\delta-1\}^{s+1} \\
& \ll_{n} & \delta^{n-s}\max\{B,\delta-1\}^{s+1}.
\end{eqnarray}
于是我们得到了想要证明的估计式\eqref{estimate in main corollary}。
\end{proof}

下面我们做一个注记,阐述关于我们为什么要选取$f(x) = x(x-1)^{n-s-1}$作为我们的计数函数。

\begin{remark}
以下的记号的意义与定理\ref{main theorem}以及推论\ref{main corollary}中保持一致。如果不考虑常数乘子的话,只有计数函数(或者说计数多项式)的次数对我们的结果有影响。事实上,设函数$f:\mathbb N^+\rightarrow \mathbb N$为任意的增的多项式函数,次数为$t+1$且满足$f(1)=0$,那么存在一个依赖于$f$的常数$C_f$使得$f(x)\leqslant C_fx(x-1)^t$。从而有
\begin{eqnarray}
\sum_{\xi\in S(X;D,B)} f(\mu_\xi(X)) & \leqslant & C_f \sum_{\xi\in S(X;D,B)} \mu_\xi(X)(\mu_\xi(X)-1)^t \nonumber \\
& \ll_f & \sum_{\xi\in S(X;D,B)} \mu_\xi(X)(\mu_\xi(X)-1)^t.
\end{eqnarray}

不久我们就会通过一个例子(例\ref{cylinder})看到,当$t\geqslant n-s-1$的时候,上式右边$\sum\limits_{\xi\in S(X;D,B)} \mu_\xi(X)(\mu_\xi(X)-1)^t$的上界估计有类似推论\ref{main corollary}中得到的估计式\eqref{estimate in main corollary}右边那样的表达式,而且各个因子的指数都是最优的。当$t<n-s-1$时,我们目前能说的只有一个平凡的估计
\begin{equation}
\sum\limits_{\xi\in S(X;B)}\mu_\xi(X)(\mu_\xi(X)-1)^t \leqslant \sum\limits_{\xi\in S(X;B)}\mu_\xi(X)(\mu_\xi(X)-1)^{n-s-1},
\end{equation}
暂时没有别的更好的结果。

因此,如果我们在上述估计中,不关心依赖于函数$f$的常数,那么选取$f(x)=x(x-1)^{n-s-1}$作为我们的计数函数对于我们来说便是足够的,正如我们在定理\ref{main theorem}中所做的那样。

设$g:\mathbb N^+\rightarrow\mathbb N$是另一个计数函数,递增,但不再满足$g(1)=0$。那么有
\begin{align}
\sum_{\xi\in S(X;D,B)} g(\mu_\xi(X)) & = g(1)N(X^{\mathrm{reg}};D,B) + \sum_{\xi\in S(X;D,B)} \left(g(\mu_\xi(X))-g(1)\right) \nonumber \\
& \leqslant g(1)N(X;D,B) + \sum_{\xi\in S(X;D,B)} \left(g(\mu_\xi(X))-g(1)\right).
\end{align}
考虑上式中的求和项$\sum\limits_{\xi\in S(X;D,B)} \left(g(\mu_\xi(X))-g(1)\right)$,它便等价于用一个满足$f(1)=0$的计数函数来对重数进行计数的问题,这是我们之前已经讨论过的。而上式中的另一项$g(1)N(X;D,B)$,如果不考虑常数$g(1)$的话,便直接是概型上代数点、有理点计数的估计这一经典的问题。关于这一问题,我们之前也已经有过了介绍以及讨论。由$X^\mathrm{reg}$与$X$是双有理等价的这一事实,我们用定理\ref{main theorem}中所选的计数函数所做出的估计是合适的。
\end{remark}

\section{一些例子}

接下来,我们给一些例子,来具体说明我们所做出过的一些论断。

\begin{example} \label{cylinder}
令$X'\hookrightarrow\mathbb P^2_{\mathbb{Q}}$一个次数等于$\delta$的既约射影平面曲线,可以由$\delta$次齐次多项式$f(T_0,T_1,T_2)$定义。我们假设$X'$有一个重数等于$\delta$的${\mathbb{Q}}$-有理点。$\delta$次齐次多项式$f(T_0,T_1,T_2)$同时也可以视作是多项式环${\mathbb{Q}}[T_0,\cdots,T_n]$中的一个$\delta$次齐次多项式。我们把$n$维射影空间$\mathbb P^n_{\mathbb{Q}}$中由$f$定义的超曲面记作$X$。不妨设平面射影曲线$X'$的奇异点射影坐标为$[1:0:0]$,那么有
\begin{align}
X^{\mathrm{sing}}(\mathbb{Q}) = & \{[x_0:\cdots:x_n] \in \mathbb{P}^n_{\mathbb{Q}}(\mathbb{Q}) \ |\ x_0:x_1:x_2 = 1:0:0 \} \cup \nonumber\\
& \{[x_0:\cdots:x_n] \in \mathbb{P}^n_{{\mathbb{Q}}}({\mathbb{Q}}) \ |\ x_0=x_1=x_2=0\},
\end{align}
并且$X^{\mathrm{sing}}(\mathbb{Q})$中所有(有理奇异)点的重数都等于$\delta$。由渐进估计式\eqref{Schanuel over Q},我们有
\begin{equation}
N(X^{\mathrm{sing}};B) = N(\mathbb{P}^{n-2}_{\mathbb{Q}};B) = \frac{2^{n-2}}{\zeta(n-1)} B^{n-1} + o(B^{n-1}), \quad B \rightarrow +\infty,
\end{equation}
从而有下列关于重数计数的渐进估计
\begin{equation*}
\sum\limits_{\xi \in S(X;B)} \mu_\xi(X)(\mu_\xi(X)-1) = \delta(\delta-1) N(X^{\mathrm{sing}};B) = O_{n} \left( \delta^2 B^{n-1} \right).
\end{equation*}
% 那么有
% \begin{equation}
% \sum\limits_{\xi \in S(X;B)} \mu_\xi(X)(\mu_\xi(X)-1) \sim_{n} \delta^2B^{n-1}.
% \end{equation}

结合这个例子我们便可以看出,当$\dim(X^\mathrm{sing}) = n-2$且$n \geqslant 3$时,推论\ref{main corollary}中估计式右边的$\delta$的阶数是最优的。更一般地,如果$X^\mathrm{sing}$包含了一个重数等于$\delta$的线性集,那么在推论\ref{main corollary}中,当$B \geqslant \delta-1$,重数的计数$\sum\limits_{\xi\in S(X;B)} \mu_\xi(X)(\mu_\xi(X)-1)$能达到$\delta$和$B$的阶数能取到(推论\ref{main corollary}估计式右边)的最大值。
\end{example}

% 当射影超曲面$X$的奇点集$X^{\mathrm{sing}}$的所有不可约分支的次数都严格大于$1$的时候,我们可以利用本文\S\ref{HB conjecture}中提到的一些结果,来得到对$\max\limits_{Z\in\mathcal Z_t}\left\{\frac{N(Z;B)}{\deg(Z)}\right\}$的比推论\ref{main corollary}中更佳的估计,从而对
% \begin{equation}
% \sum_{\xi\in S(X;B)}\mu_\xi(X)(\mu_\xi(X)-1)^{n-s-1}
% \end{equation}
% 给出更优的估计。

设$K$为一个一般的数域,$X \hookrightarrow \mathbb{P}^n_K$为任意一个次数等于$\delta$的超曲面。想要利用定理\ref{main theorem}来解决这种更加一般的重数计数问题,关键在于给出定理估计式\eqref{estimate in main theorem}中的项
\begin{equation}
\max_{Z \in \mathcal{Z}_t} \left\{ \frac{N(Z;D,B)}{\deg(Z)} \right\}
\end{equation}
的一致的估计,其中$Z \in \mathcal{Z}_t$,$t = 1,\ldots,s$,$\mathcal{Z}_t$仍是由$X$出发构造的相交树的特定的顶点之集,具体定义可以回顾定义\ref{def of Z_s}。

在定理\ref{refined Schanuel estimate for rational points}中,我们已经仔细考察了$K = \mathbb{Q}$的情形,并且由此定理证明了推论\ref{main corollary}中的结论。如果$X$的奇点集$X^{\mathrm{sing}}$的所有不可约分支的次数都是严格大于$1$的,那么我们就可以利用我们在\S\ref{HB conjecture}中介绍的一些结果来得到关于
\begin{equation}
\sum_{\xi \in S(X;B)} \mu_\xi(X)(\mu_\xi(X)-1)^{n-s-1}
\end{equation}
比在推论\ref{main corollary}中更加精细的一些估计。其原因在于,在这种情况下,对于
\begin{equation}
\max\limits_{Z\in\mathcal{Z}_t} \left\{ \frac{N(Z;B)}{\deg(Z)} \right\},
\end{equation}
我们有比定理\ref{refined Schanuel estimate for rational points}更优的估计。

\begin{example} \label{example using HB conj}
设$\delta \geqslant 3$以及$n \geqslant 2$为正整数,令
\begin{equation} \label{hypersurface Z}
\begin{tikzcd}
Z \arrow[r, hookrightarrow] & \mathbb{P}^{n+2}_{\mathbb{Q}} = \proj\left(\mathbb{Q}[X,Y,T_0,\cdots,T_n]\right)
\end{tikzcd}
\end{equation}
是一个由$\delta$次齐次多项式
\begin{equation} \label{hypersurface Z2}
F(X,Y,T_0,\cdots,T_n) = Y^\delta + Xf(T_0,T_1,\cdots,T_n)
\end{equation}
定义的超曲面,其中$f(T_0,\cdots,T_n)$是一个$\delta-1$次齐次多项式。我们假设$f$在$P^{n}$中定义了一个光滑的超曲面,我们把它记作$Z'$。那么多项式$F(X,Y,T_0,\cdots,T_n)$是不可约的。由参考文献~\inlinecite{LiuQing}的习题2.4.1可知,超曲面$Z$是整的。由参考文献~\inlinecite{Browning_Heath06II}的 Theorem 1,当$\delta \geqslant 3$且$n \geqslant 2$时,对任意的$\epsilon>0$,估计式
\begin{equation} \label{N(Z';B)}
N(Z';B) \ll_{K,n,\delta,\epsilon} B^{n-1+\epsilon}.
\end{equation}
对所有的这样的光滑超曲面$Z'$都是成立的。

与此同时,由Jacobian判别准则(详见参考文献~\inlinecite{LiuQing}第4章的Theorem 2.19),超曲面$Z$的奇点集由下列方程组给出
\begin{equation}
\begin{cases}
\delta Y^{\delta-1} = 0 \\
f(T_0,\cdots,T_n) = 0 \\
X \frac{\partial f}{\partial T_0} = 0 \\
\phantom{hehe} \vdots \\
X \frac{\partial f}{\partial T_n} = 0
\end{cases}
\end{equation}
% $$0 = \delta Y^{\delta-1} = f(T_0,\cdots,T_n) = X\frac{\partial f}{\partial T_0} = \cdots = X\frac{\partial f}{\partial T_n}.$$
由于超曲面$Z'$在$\mathbb{Q}$上是光滑的,因此多项式$f, \frac{\partial f}{\partial T_0}, \cdots, \frac{\partial f}{\partial T_n}$没有公共零点。所以,任取超曲面$Z$上的奇异点$\xi\in Z^{\mathrm{sing}}(\overline{\mathbb{Q}})$,其射影坐标,设为$[x:y:t_0:\cdots:t_n]$,必须满足如下条件
\begin{equation}
\begin{cases}
x = y = 0 \\
[t_0:\cdots:t_n] \in Z'(\overline{\mathbb{Q}})
\end{cases}
\end{equation}
% $$X=Y=0, ~~ \text{ 且 } ~~ [T_0:\cdots:T_n] \in X'(\overline K).$$
容易看出,这些奇异点在超曲面$Z$中的重数都是$2$。由定义,超曲面$Z$的奇点集$Z^\mathrm{sing}$在$Z$中的余维数等于$2$,也就是说它的维数等于$n-1$。

我们把在定理\ref{main theorem}中考察过的问题应用到本例中来。具体来说,我们要考察和式$\sum\limits_{\xi \in S(Z;B)} \mu_{\xi}(Z)(\mu_{\xi}(Z)-1)^2$。对于本例来说,首先,对于任意的正整数$\delta \geqslant 3$和$n \geqslant 2$,以及任意的正实数$\epsilon > 0$,所用通过\eqref{hypersurface Z},\eqref{hypersurface Z2}中的方法定义的超曲面$Z,Z'$,我们都有
\begin{equation}
\sum_{\xi \in S(Z;B)} \mu_{\xi}(Z)(\mu_{\xi}(Z)-1)^2 = 2N(Z';B).
\end{equation}
假设我们依照\S\ref{construction of intersection trees}中的程序来构造(一族)相交树的话,我们会发现,我们只能得到一颗相交树,而且这个相交树里面只有根节点,即$Z$,$V \left(\frac{\partial F}{\partial X}\right)$,$V \left(\frac{\partial F}{\partial Y}\right)$的(正常)相交积,这个根节点没有任何的子孙。这是一种退化了的情况。把定理\ref{main theorem}的结果直接套用在这个例子上,我们会有
\begin{equation}
\sum_{\xi \in S(Z;B)} \mu_{\xi}(Z)(\mu_{\xi}(Z)-1)^2 \leqslant \delta(\delta-1)^2 \frac{N(Z';B)}{\delta-1} = \delta(\delta-1)N(Z';B).
\end{equation}
这正好达到了我们在定理\ref{main theorem}中给出的估计的上界。

% 而根据参考文献~\inlinecite{Browning_Heath06II}中 Theorem 1,我们
% \begin{equation}
% \sum_{\xi \in S(X;B)} \mu_{\xi}(X)(\mu_{\xi}(X)-1)^2 = 2N(X;B) \ll_{K,n,\delta,\epsilon} B^{n-1+\epsilon},
% \end{equation}
% 对任意的$\epsilon>0$成立。这个例子表明,在定理\ref{main theorem}的估计式\eqref{estimate in main theorem}中的上界是能够达到的。

而根据参考文献~\inlinecite{Browning_Heath06II}的 Theorem 1给出的非奇异超曲面上有理点的估计式\eqref{N(Z';B)},对于所用通过\eqref{hypersurface Z},\eqref{hypersurface Z2}中的方法定义的超曲面$Z, Z'$,我们都有估计
\begin{equation}
\sum_{\xi \in S(Z;B)} \mu_{\xi}(Z)(\mu_{\xi}(Z)-1)^2 = 2N(Z';B) \ll_{n,\delta,\epsilon} B^{n-1+\epsilon}.
\end{equation}
相对于我们在推论\ref{main corollary}中给出的估计,上述估计式给出了对于$B$的阶数更优的估计。然而在这个估计中,并没有对超曲面次数$\delta$在估计式中的阶进行讨论,而且就作者目前的知识而言,通过这种方法在相关的估计中还无法对$\delta$的阶进行控制。

% 并没有对超曲面次数$\delta$在估计式中的阶进行讨论,就作者目前的知识而言,本文在相关的估计中还无法对$\delta$的阶进行控制。
\end{example}

% 至于一般的代数点的重数计数问题,如果相应的次数,维数满足\S\ref{counting algebraic points in Pn}中列出来的那些条件,那么仿照证明推论\ref{main corollary}的方法,我们也能够由定理 \ref{refined Schanuel estimate for rational points}得到类似的一些估计。

\section{问题的推广}
本文所解决的问题是针对数域上射影超曲面的,我们也可以在一个一般的射影算术簇$X$上考虑上面的代数点、有理点的重数计数的问题。但是情况会复杂很多。

仿照参考文献~\inlinecite{Liu-multiplicity}中提出的猜想,本文也提出类似的猜想

\begin{conjecture}
令$X$是$n$维射影空间$\mathbb P^n_K$的一个既约的纯维数的闭子概型。令$X$的维数等于$d$,次数等于$\delta$,假设$X$的奇点集$X^{\mathrm{sing}}$的维数等于$s$。那么有
\begin{equation}
\sum_{\xi\in S(X;B)} \mu_\xi(X)(\mu_\xi(X)-1)^{d-s} \ll_{n,K} \delta^{d-s+1}B^{s+1}.
\end{equation}
\end{conjecture}
