\thusetup{
  %******************************
  % 注意:
  %   1. 配置里面不要出现空行
  %   2. 不需要的配置信息可以删除
  %******************************
  %
  %=====
  % 秘级
  %=====
%   secretlevel={秘密},
%   secretyear={10},
  %
  %=========
  % 中文信息
  %=========
  ctitle={定义在数域上的超曲面的\\重数计数问题},
  cdegree={理学博士},
  cdepartment={数学科学系},
  cmajor={数学},
  cauthor={文豪},
  csupervisor={张贺春教授},
%   cassosupervisor={陈文光教授}, % 副指导老师
%   ccosupervisor={某某某教授}, % 联合指导老师
  % 日期自动使用当前时间,若需指定按如下方式修改:
  cdate={二〇一七年十二月},
  %
  % 博士后专有部分
%   cfirstdiscipline={计算机科学与技术},
%   cseconddiscipline={系统结构},
%   postdoctordate={2009年7月——2011年7月},
%   id={编号}, % 可以留空: id={},
%   udc={UDC}, % 可以留空
%   catalognumber={分类号}, % 可以留空
  %
  %=========
  % 英文信息
  %=========
  etitle={Counting Multiplicities in a Hypersurface over Number Fields},
  % 这块比较复杂,需要分情况讨论:
  % 1. 学术型硕士
  %    edegree:必须为Master of Arts或Master of Science(注意大小写)
  %             “哲学、文学、历史学、法学、教育学、艺术学门类,公共管理学科
  %              填写Master of Arts,其它填写Master of Science”
  %    emajor:“获得一级学科授权的学科填写一级学科名称,其它填写二级学科名称”
  % 2. 专业型硕士
  %    edegree:“填写专业学位英文名称全称”
  %    emajor:“工程硕士填写工程领域,其它专业学位不填写此项”
  % 3. 学术型博士
  %    edegree:Doctor of Philosophy(注意大小写)
  %    emajor:“获得一级学科授权的学科填写一级学科名称,其它填写二级学科名称”
  % 4. 专业型博士
  %    edegree:“填写专业学位英文名称全称”
  %    emajor:不填写此项
  edegree={Doctor of Philosophy},
  emajor={Mathematics},
  eauthor={Wen Hao},
  esupervisor={Professor Zhang Hechun},
%   eassosupervisor={Chen Wenguang},
  % 日期自动生成,若需指定按如下方式修改:
  edate={December, 2017}
  %
  % 关键词用“英文逗号”分割
  ckeywords={重数计数, 射影超曲面, 高度函数, 一般化的 Schanuel 估计, 相交树},
  ekeywords={counting multiplicity, projective hypersurface, height function, generalized Schanuel's estimate, intersection tree}
}

% 定义中英文摘要和关键字
\begin{cabstract}
考虑定义在数域上的射影超曲面,我们可以通过局部 Hilbert-Samuel 函数定义其上点的重数。取定一个计数函数,我们考虑在这个射影超曲面上代数点的重数的计数问题:即对超曲面上,高度不超过定值,定义域相对于基域的扩张次数等于一个定值的所有代数点,用取定的计数函数作用在这些点的重数上,求和。本文将对这个问题给出一个上界估计,这个上界将会与射影超曲面的次数,奇点集的维数,高度的上界,以及相应域扩张次数有关。这个估计可以在某种程度上衡量射影超曲面的奇点集的复杂程度。
\end{cabstract}

% 如果习惯关键字跟在摘要文字后面,可以用直接命令来设置,如下:
% \ckeywords{\TeX, \LaTeX, CJK, 模板, 论文}

\begin{eabstract}
We fix a counting function of multiplicities of algebraic points in a projective hypersurface over a number field, and take the sum over all algebraic points of bounded height and fixed degree. An upper bound for the sum with respect to this counting function will be given in terms of the degree of the hypersurface, the dimension of the singular locus, the upper bounds of height, and the degree of the field of definition. This upper bound gives a description of the complexity of the singular locus of this hypersurface to some extent.
\end{eabstract}

% \ekeywords{\TeX, \LaTeX, CJK, template, thesis}
