\chapter{数论的一些基础知识}
\label{apdx: number theory}
在这个附录中,我们将介绍一些本文需要用到的数论的一些知识,并对正文中简要提及的内容做一定的补充。主要的参考文献是~\inlinecite{YinLinsheng},~\inlinecite{LiJinghui}以及~\inlinecite{LangANT}。
\section{域的有限扩张的 Galois 理论}
\label{apdx: galois theory}
设$K$为域,$L/K$为有限扩域,也就是说作为$K$-线性空间,$L$是有限维的。此维数被记作$[L:K]$,被称为$L/K$的扩张次数。称$L$的一个自同构$\sigma: L \to L$为一个$K$-自同构,如果$\sigma|_K = \identity_K$,即$\sigma$限制在$K$上是恒等映射。记$L$所有的$K$-自同构组成的集合为$\aut_K(L)$。以映射的复合作为乘积运算,则$\aut_K(L)$构成一个群。一般地,$\#\aut_K(L) \leqslant [L:K]$。如果等号成立,则称$L/K$为 Galois 扩张,而把$\aut_K(L)$记为$\gal(L/K)$,称为域扩张$L/K$的 Galois 群。

一个更一般的,对无限扩张也适用的定义是
\begin{definition}
设$L/K$是域的代数扩张。若它是可分且正规的,那么称域扩张$L/K$为 Galois 扩张。
\end{definition}
这里,$L/K$是代数扩张指的是,对$L$中任一个元素$a$,都存在多项式$f_a(X) \in K[X]$使得$f_a(a)=0$。记这些多项式中次数最低的首一的多项式(从而必然不可约)为$m_a(X)$,称为$a$的极小多项式。代数扩张$L/K$是可分的指的是,对$L$中任一个元素$a$,其极小多项式$m_a(X)$无重根;代数扩张$L/K$是正规的,指的是$L$包含$a$的所有共轭元,即$m_a(X)$的所有根。

有限 Galois 扩张的 Galois 群与中间域有如下关系,被称为有限 Galois 扩张的主定理。
\begin{theorem}[\inlinecite{YinLinsheng}, 定理 B.3]
\label{galois main theorem}
设$L/K$为域的有限 Galois 扩张,并令$G = \gal(L/K)$。那么有集合间的双射
\begin{equation}
\{ \text{使$K\subseteq M \subseteq L$的域$M$} \} \overset{1:1}{\longleftrightarrow} \{ \text{$G$的子群$H$} \}
\end{equation}
其中
\begin{gather*}
H = \{ \sigma \in G \ |\ \text{对所有的$x\in M$使得$\sigma(x) = x$} \}, \\
M = \{ x\in L \ |\ \text{对所有的$\sigma\in H$使得$\sigma(x) = x$} \}.
\end{gather*}
\end{theorem}

关于域的有限扩张,有两个基本而且重要的映射:迹映射与范映射。

\begin{definition}
设$L/K$为域的有限扩张,$a\in L$,考虑$K$-线性映射
$$L \longrightarrow L: x\mapsto ax.$$
分别定义$N_{L/K}(a)$与$\tr_{L/K}(a)$为以上线性映射的行列式与迹。
\end{definition}

迹与范的基本性质可以列举如下

\begin{proposition}[\inlinecite{YinLinsheng}, B.15]
设$L/K$为域的有限扩张,那么
\begin{enumerate}
\item 对于$a,b\in L$
\begin{gather}
N_{L/K}(ab) = N_{L/K}(a)\cdot N_{L/K}(b), \\
\tr_{L/K}(a+b) = \tr_{L/K}(a) + \tr_{L/K}(b).
\end{gather}
\item 令$n = [L:K]$,若$a\in K$,那么
\begin{gather}
N_{L/K}(a) = a^n, \\
\tr_{L/K}(a) = na.
\end{gather}
\item 如果$L/K$为 Galois 扩张,$G = \gal(L/K)$,则
\begin{gather}
N_{L/K}(a) = \prod\limits_{\sigma\in G} \sigma(a), \\
\tr_{L/K}(a) = \sum\limits_{\sigma\in G} \sigma(a).
\end{gather}
\end{enumerate}
\end{proposition}

\section{数域的基本性质}
以下设$K$为数域,即有理数域$\mathbb{Q}$的一个有限扩域。称$a\in K$为代数整数,如果$a$是一个首一的整系数多项式的根。数域$K$中所有代数整数组成一个整环,通常记为$\mathcal{O}_K$。或者说,$\mathcal{O}_K$是有理整数环$\mathbb{Z}$在域$K$里面的整闭包。一般地,对于环的扩张$B/A$,称$b\in B$在$A$上是整的,如果$b$是某个首一的$A$系数多项式的根。$B$在$A$上是整的元素组成的集合被称为$B$在$A$中的整闭包。


一个重要事实是,$\mathcal{O}_K$总是一个 Dedekind 环,也就是说$\mathcal{O}_K$总满足以下条件
\begin{itemize}
\item $\mathcal{O}_K$为 Noether 环;
\item $\mathcal{O}_K$是整闭整环;
\item $\mathcal{O}_K$中非零素理想都是极大理想。
\end{itemize}
或者等价地,$\mathcal{O}_K$中非零理想总有唯一的素理想分解。这是因为有理整数环$\mathbb{Z}$本身是一个 Dedekind 环,再加上
\begin{theorem}
设$A$是一个 Dedekind 环,其分式域为$K$。令$L/K$为一个有限的域扩张$B$为$A$在$L$中的整闭包。那么$B$也是一个Dedekind环。
\end{theorem}

在$\mathcal{O}_K$中考虑比非零理想更广一点的分式理想,即$K$的子集合$\mathfrak{a}$,使得存在$\mathcal{O}_K$中非零元$c$,使得$c\mathfrak{a}$为$\mathcal{O}_K$的非零理想。于是唯一素理想分解定理可以表述为
\begin{theorem}[\inlinecite{YinLinsheng}定理 4.12]
设$K$为数域,$\mathfrak{a}$为分式理想,那么$\mathfrak{a}$可以唯一表示为
\begin{equation}
\mathfrak{a} = \prod\limits_{\mathfrak{p}} \mathfrak{p}^{e_{\mathfrak{p}}},
\end{equation}
其中$\mathfrak{p}$遍历$\mathcal{O}_K$的所有非零素理想,$e_{\mathfrak{p}}\in\mathbb{Z}$,且除有限个$\mathfrak{p}$外有$e_{\mathfrak{p}} = 0$。
\end{theorem}

$\mathcal{O}_K$的全体分式理想关于分式理想的乘法构成一个群。这个群里有一个特殊的子群,由分式主理想,即形如$a\cdot\mathcal{O}_K$,$a\in K^{\times}$,的分式理想,构成的子群。

\begin{definition}
数域$K$的理想类群指的是$\mathcal{O}_K$的全体分式理想构成的群,模掉全体分式主理想构成的子群,所得的商群,被记作$\operatorname{Cl}(K)$。
\end{definition}

关于数域$K$还有一个重要的群,就是它的单位群,依定义等于$\mathcal{O}_K$中所有可逆元构成的乘法群$\mathcal{O}_K^{\times}$。代数数论两大基本定理就是关于数域的理想类群以及单位群的,叙述如下:

\begin{theorem}[类数有限定理]
数域的理想类群是有限群。
\end{theorem}
数域$K$的理想类群的元素个数被称作$K$的类数,通常记作$h_K$。

\begin{theorem}[Dirichlet 单位定理]
数域的单位群是有限生成的 Abel 群。
\end{theorem}
数域$K$的单位群$\mathcal{O}_K^{\times}$的自由部分同构于$\mathbb{Z}^r$,其秩$r = r_1 + r_2 - 1,$ 其中$r_1, r_2$分别为数域$K$的实位点个数与复位点个数(定义详见正文部分的\S \ref{abs values on number fields}),扭部分由$K$中单位根组成,元素个数通常被记作$\omega_K$。

接下来再介绍两个与数域$K$相关的重要的数量。
\begin{definition} \label{number field discriminant}
设$K$为数域,其代数整数环$\mathcal{O}_K$作为$\mathbb{Z}$-模设有基底$a_1,\cdots,a_n$,$n = [K:\mathbb{Q}]$。称
\begin{equation}
D_K := \det(\tr_{K/\mathbb{Q}}(a_ia_j)_{1\leqslant i,j \leqslant n})
\end{equation}
为数域$K$的判别式。
\end{definition}
判别式$D_K$的定义与整基$a_1,\cdots,a_n$的选取无关。$D_K$在某种程度上测量了数域$K$的整数环$\mathcal{O}_K$的``大小'',并且掌握了素数的在域扩张$K/\mathbb{Q}$下的分歧情况。

\begin{definition} \label{number field regulator}
设数域$K$的单位群$\mathcal{O}_K^{\times}$秩为$r$,$u_1, \cdots, u_r$为其自由部分的一组生成元。设$\sigma_1,\cdots,\sigma_{r+1}$为数域$K$的$r+1$个无限位点。那么
\begin{equation}
R_K := \left|\det(\log\|u_i\|_{\sigma_j})_{1\leqslant i,j \leqslant r}\right|
\end{equation}
与$u_i$以及$\sigma_i$的选取无关,被称为数域$K$的调控子,或称导子。
\end{definition}
数域$K$的调控子$R_K$反映的是数域$K$中单位的``密度''大小。

\section{位点、完备化与数域扩张}
关于数域的位点以及在位点处的完备化,我们已经在正文的\S \ref{abs values on number fields}具体叙述过了,这里就不再重复了,只回忆一下相关的记号。数域$K$的位点的集合我们记为
\begin{equation}
M_K = M_{K,f} \bigsqcup M_{K,\infty},
\end{equation}
其中$M_{K,f}$由$K$的非零素理想组成,$M_{K,\infty}$由一些$K$到$\mathbb{R}$或$\mathbb{C}$的域嵌入组成。数域在位点$v\in M_K$处的完备化被记作$K_v$。$K_v$有可能是实数域$\mathbb{R}$或复数域$\mathbb{C}$或$p$-进数域$\mathbb{Q}_p$的有限扩张,$p$为有理素数。$v$为有限位点的时候,$K_v$的整数环为
\begin{equation}
\mathcal{O}_{v} = \left\{ a\in K_v \ \middle|\ \ord_v(a) \geqslant 0 \right\},
\end{equation}
它是一个局部环,有极大理想
\begin{equation}
\mathfrak{m}_{v} = \left\{ a\in K_v \ \middle|\ \ord_v(a) \geqslant 1 \right\}.
\end{equation}
$\mathfrak{m}_{v}$由一个元素$\pi$生成。$\pi$满足$\ord_v(\pi) = 0$,被称作是归一化子。一般来说$K_v$都不是代数闭的,除了当$v$是复位点的时候。考虑$v$为有限位点的情况,将$K_v$取代数闭包$\overline{K_v}$,则$K_v$上的$v$-进绝对值自然延拓到$\overline{K_v}$上。但此时$\overline{K_v}$关于这个绝对值并不完备。再对域$\overline{K_v}$做完备化,得到的域
\begin{equation}
\mathbb{C}_v = \widehat{\overline{K_v}}
\end{equation}
是一个完备的代数闭域,与复数域$\mathbb{C}$同构。

下面我们介绍数域的位点在数域扩张中的性态。设$L/K$为数域的扩张,相应的代数整数环分别为$\mathcal{O}_L$与$\mathcal{O}_K$。任取$\mathfrak{p}$为$\mathcal{O}_K$的非零素理想。那么$\mathfrak{p}\mathcal{O}_L$是$\mathcal{O}_L$的一个非平凡的理想,有分解
\begin{equation} \label{prime ideals decomposition}
\mathfrak{p}\mathcal{O}_L = \mathfrak{P}_1^{e_1}\cdots \mathfrak{P}_r^{e_r}.
\end{equation}
称素理想$\mathfrak{p}$在素理想$\mathfrak{P}_i$之下,也称素理想$\mathfrak{P}_i$在素理想$\mathfrak{p}$之上。

\begin{definition}\ 
\begin{enumerate}
\item $e_i = e(\mathfrak{P}_i|\mathfrak{p})$被称为$\mathfrak{P}_i$关于$\mathfrak{p}$的分歧指数。
\item 有剩余域的扩张$\mathcal{O}_K/\mathfrak{p} \subseteq \mathcal{O}_L/\mathfrak{P}_i$,记
\begin{equation}
f_i = f(\mathfrak{P}_i|\mathfrak{p}) = [\mathcal{O}_L/\mathfrak{P}_i: \mathcal{O}_K/\mathfrak{p}],
\end{equation}
$f_i$被称作$\mathfrak{P}_i$关于$\mathfrak{p}$的剩余次数。
\end{enumerate}
\end{definition}

对于数域的扩张$L/K$,设素理想$\mathfrak{p}\in M_{K,f}$有分解\eqref{prime ideals decomposition},那么有数量关系
\begin{equation}
[L_{\mathfrak{P}_i} : K_{\mathfrak{p}}] = e(\mathfrak{P}_i|\mathfrak{p})\cdot f(\mathfrak{P}_i|\mathfrak{p}),
\end{equation}
和正文中提到的次数公式\eqref{degree formula}:
\begin{equation}
[L: K] = \sum\limits_{i=1}^r e(\mathfrak{P}_i|\mathfrak{p}) \cdot f(\mathfrak{P}_i|\mathfrak{p}).
\end {equation}
于是式\eqref{explicit v-adic abs value}也可以被写作
\begin{equation}
|a|_v = p^{-\frac{\ord_v(a)}{e(v|p)}},
\end{equation}
其中$p$为$v\in M_{K,f}$之下的有理素数。以及
\begin{gather}
N_{L/K}(\cdot) = \prod\limits_{w|v} N_{L_w/K_v}(\cdot), \\
\tr_{L/K}(\cdot) = \sum\limits_{w|v} \tr_{L_w/K_v}(\cdot)
\end{gather}

\section{$\zeta$函数}
\label{apdx: zeta function}
\begin{definition}
数域$K$的 Dedekind $\zeta$函数被定义作
\begin{equation} \label{zeta_K}
\zeta_K(s) = \sum\limits_{\mathfrak{a}} \dfrac{1}{N\mathfrak{a}^s} = \prod\limits_{\mathfrak{p}} \dfrac{1}{N\mathfrak{p}^s},
\end{equation}
其中$\mathfrak{a}$遍历$\mathcal{O}_K$的非零理想,$\mathfrak{p}$遍历$\mathcal{O}_K$的非零素理想。定义式\eqref{zeta_K}中的级数在$\realpart(s) > 1$时绝对收敛。
\end{definition}

\begin{definition}
数域$K$的完备 Dedekind $\zeta$函数被定义作
\begin{equation} \label{complete zeta_K}
\widehat{\zeta}_K(s) = |D_K|^{\frac{s}{2}}\Gamma_{\mathbb{R}}(s)^{r_1}\Gamma_{\mathbb{C}}(s)^{r_2} \zeta_K(s),
\end{equation}
其中$D_K$是数域$K$的判别式(见定义\ref{number field discriminant}),$r_1$为数域$K$实位点的个数,$r_2$为数域$K$复位点的个数,$\Gamma_{\mathbb{R}}(s) = \pi^{-\frac{s}{2}} \Gamma(\frac{s}{2})$,$\Gamma_{\mathbb{C}}(s) = 2(2\pi)^{-s} \Gamma(s)$,$\Gamma(s)$为 Gamma 函数,定义如下
\begin{equation}
\Gamma(s) = \int\limits_0^{\infty} e^{-t}t^s\dfrac{dt}{t}
\end{equation}
\end{definition}

\begin{theorem}[\inlinecite{YinLinsheng}定理7.10]
设$K$为数域,令$K$的类数为$h_K$,调控子为$R_K$,单位根个数为$\omega_K$,实位点个数为$r_1$,复位点个数为$r_2$,记$r = r_1+r_2-1$。那么
\begin{enumerate}
\item $\zeta_K(s)$可解析延拓为整个复平面上的亚纯函数,在$s=1$处有一阶极点外,在$s=1$之外全纯。
\item 有函数方程$\widehat{\zeta}_K(s) = \widehat{\zeta}_K(1-s)$。
\item $\lim\limits_{s\to 1} (s-1)\zeta_K(s) = \dfrac{2^{r_1}(2\pi)^{r_2}h_KR_K}{\omega_K|D_K|^{\frac{1}{2}}}$。
\item $\lim\limits_{s\to 0} s^{-1}\zeta_K(s) = -\dfrac{h_KR_K}{\omega_K}$。
\end{enumerate}
\end{theorem}

\section{加元环与理元群}
\label{apdx: adele ring and idele group}
数域$K$在位点$v\in M_K$处的局部化域$K_v$关于绝对值$|\cdot|_v$都是局部紧域,而且当$v\in M_{K,f}$为有限位点的时候,有紧开子群$\mathcal{O}_v$。考虑限制直积(限制直积用$\sideset{}{'}\prod$表示)
\begin{equation} \label{adele ring}
\mathbb{A}_K := \sideset{}{'}\prod_{v\in M_K} K_v \\
= \left\{ (a_v)_{v} \in \prod_{v\in M_K} K_v \ \middle|\ \text{除有限个位点$v$外有$a_v \in \mathcal{O}_v$} \right\}.
\end{equation}

\begin{definition}
式\eqref{adele ring}给出的环$\mathbb{A}_K$被称作数域$K$的加元环。
\end{definition}

容易看出,数域$K$可以自然地看作是他的加元环$\mathbb{A}_K$的子域。

加元环$\mathbb{A}_K$被赋予如下的拓扑而成为一个(关于加法)局部紧的拓扑环:对于$M_K$的包含$M_{K,\infty}$的有限子集$T$,考虑$\mathbb{A}_K = \sideset{}{'}\prod\limits_{v\in M_K} K_v$的子集
\begin{equation}
\mathbb{A}(T) := \prod\limits_{v\in T} K_v \times \prod\limits_{v\in M_K\setminus T} \mathcal{O}_v,
\end{equation}
并在其上赋予直积拓扑。称$V\subseteq \mathbb{A}_K$为开集,如果对所有的$T$,$V\cap \mathbb{A}(T)$为$\mathbb{A}(T)$中的开集。

\begin{proposition} [\inlinecite{YinLinsheng}, 命题 6.78]
$K$在$\mathbb{A}_K$中离散,并且$\mathbb{A}_K / K$为紧群。
\end{proposition}

在每个局部紧域$K_v$上,有关于加法不变的 Haar 测度。对每个$v\in M_K$,在$K_v$上取定一个 Haar 测度$\alpha_v$,要求除有限个$v$外$\alpha_v(\mathcal{O}_v) = 1$,那么在每个$\mathbb{A}(T)$上的积测度
\begin{equation}
\alpha = \prod\limits_{v\in M_K} \alpha_v
\end{equation}
是良定义的,而且可以扩展为加元环$\mathbb{A}_K$上的一个 Haar 测度。

\begin{proposition} [\inlinecite{LiJinghui}, 命题 12.9] \ 
\begin{enumerate}
\item 设$\widehat{\alpha}_v$为$\alpha_v$的对偶测度。则除有限个$v$外$\widehat{\alpha}_v(\mathcal{O}_v) = 1$,并且$\prod\limits_{v\in M_K} \widehat{\alpha}_v$为$\alpha = \prod\limits_{v\in M_K} \alpha_v$的对偶测度。
\item 如果$\alpha, \alpha_v$为自对偶测度,那么$\alpha(\mathbb{A}_K / K) = 1$。
\end{enumerate}
\end{proposition}

\begin{definition}
若$\mathbb{A}_K$的测度$\alpha$满足$\alpha(\mathbb{A}_K / K) = 1$,则称$\alpha$为玉河测度。
\end{definition}

加元环$\mathbb{A}_K$的全体可逆元构成一个群$\mathbb{A}_K^\times$:
\begin{equation} \label{idele group}
\mathbb{A}_K^\times := \sideset{}{'}\prod_{v\in M_K} K_v^\times \\
= \left\{ (a_v)_{v} \in \prod_{v\in M_K} K_v^\times \ \middle|\ \text{除有限个位点$v$外有$a_v \in \mathcal{O}_v^\times$} \right\}.
\end{equation}

\begin{definition}
式\eqref{idele group}给出的群$\mathbb{A}_K^\times$被称为数域$K$的理元群。
\end{definition}
有$K^\times \subseteq \mathbb{A}_K^\times$,而且$\mathbb{A}_K^\times$也是局部紧的拓扑群。

\begin{proposition} [\inlinecite{LiJinghui}, 命题 12.11]
设$v\in M_{K,f}$为有限位点,$\alpha_v$是$K_v$上的一个 Haar 测度,那么
\begin{equation}
d\mu_v(x) = \dfrac{d\alpha_v(x)}{\|x\|_v}
\end{equation}
是$K_v^\times$的 Haar 测度,并且有
\begin{equation}
\mu_v(\mathcal{O}_v^\times) = (1-\dfrac{1}{Nv})\alpha_v(\mathcal{O}_v).
\end{equation}
\end{proposition}


数论本身是一个非常大的题目,因为篇幅关系,有关本文需要用到或者了解的一些数论的基础知识就介绍到这里。更加深入的有关数论的知识可以参阅本章开头列出的著作,以及没有列出来的其他的著作。

\chapter{代数几何的一些基础知识}
\label{apdx: algebraic geometry}
本章附录主要简要地收录一些和正文相关的代数几何方面的基础知识,特别是向量丛,除子与射影概型相应的概念一及它们之间关系。主要的参考文献是~\inlinecite{GTM52},~\inlinecite{FuLei},~\inlinecite{LiuQing}以及~\inlinecite{Fulton}的附录B。

关于预层,层,概型,概型的态射这些最基本的概念及其性质,因为篇幅关系这里就不详细回忆了。
% 设$X$为一个拓扑空间。$X$上的集合的预层$\mathcal{F}$由如下的内容组成:
% \begin{enumerate}
% \item $X$的每个非空开子集$U$对应一个集合$\mathcal{F}(U)$,里面元素被称为$\mathcal{F}$在$U$上的截影。
% \item 对$X$的开子集的每个包含关系$V\subseteq U$,有限制映射$\rho_{UV}: \mathcal{F}(U) \to \mathcal{F}(V)$,满足
% \begin{itemize}
% \item $\rho_{UU} = \identity_U$。
% \item 如果$W\subseteq V\subseteq U$
% \end{itemize}
% \end{enumerate}

% 本文中所说的域$k$上的簇$X$,指的是$X$是整的(既约的且不可约的),在$\spec(k)$上是有限型的概型。以下,
设$X$为域$k$上的概型,$\mathcal{O}_X$为$X$的结构层。

\begin{definition}
概型$X$上的秩为$r$的向量丛$E$是$X$上的一个概型$\pi: E \to X$,以及$X$的开覆盖$\{U_i\}$和同构$\varphi_i: \pi^{-1}(U_i) \to U_i\times \mathbb{A}^r$,使得在$U_i\cap U_j$上,$\varphi_i\circ\varphi_j^{-1}$是线性的,即态射
\begin{equation}
\varphi_{ij}: U_i\cap U_j \to \gl(r,k),
\end{equation}
其中$\gl(r,k)$是仿射概型$\spec(k[S,T_{ij}\ |\ 1\leqslant i,j \leqslant r] / (S\cdot\det(T_{ij}) - 1))$。
\end{definition}

与之等价的一个概念是
\begin{definition}
设$\mathscr{E}$是一个$\mathcal{O}_X$-模。它被称为是局部自由的,如果存在$X$的开覆盖$\{U_i\}$使得每个$\mathscr{E}|_{U_i}$都是一个自由$\mathcal{O}_{U_i}$-模,即$\mathscr{E}|_{U_i}\cong \mathcal{O}_{U_i}^{\oplus r}, r\in \mathbb{N}^+$。
\end{definition}

$X$上的秩为$r$的向量丛的同构类与$X$上的秩为$r$的局部自由$\mathcal{O}_X$-模的同构类之间的一一对应关系如下
$$
\begin{tikzcd}[row sep = tiny]
\{ \text{$X$上的秩为$r$的向量丛} \} / \cong \arrow[r, leftrightarrow] & \{ \text{$X$上的秩为$r$的局部自由$\mathcal{O}_X$-模} \} / \cong \\
\phantom{hehehe} E \arrow[r, mapsto] & \text{$E$的截影层$\mathscr{E}$:} U \mapsto \{s: U \to E \ |\ \pi\circ s = \identity_U \} \\
\SPEC(\sym(\mathscr{E}^{\vee})) & \mathscr{E} \arrow[l, mapsto]
\end{tikzcd}
$$
其中$\mathscr{E}^{\vee} = \sheafhom_{\mathcal{O}_X}(\mathscr{E}, \mathcal{O}_X)$,$\sym$表示对称代数,也是一个秩为$r$的局部自由模。于是,如果不产生混淆,可以随意交替、混杂使用``向量丛''与``局部自由模''的概念与记号。

特别地,当秩$r$等于$1$时,称向量丛$E$为线丛,或称局部自由模$\mathscr{E}$为可逆层。此时,通常换用记号$\mathscr{L}$表示一个可逆层。由于有典范的同构
\begin{equation}
\mathscr{L}\otimes_{\mathcal{O}_X} \sheafhom_{\mathcal{O}_X}(\mathscr{L}, \mathcal{O}_X) \cong \mathcal{O}_X,
\end{equation}
概型$X$上的可逆层的同构类在张量积$\otimes$运算下称为一个群,被称为$X$的 Picard 群,记作$\Pic(X)$。

% 考虑射影空间$\mathbb{P}_k^n = \proj k[T_0,\cdots,T_n]$上一个特殊的可逆层,Serre 扭层$\mathcal{O}_{\mathbb{P}_k^n}(1)$。记分次环$S = k[T_0,\cdots,T_n] = \bigoplus\limits_{d\geqslant0}S_d$。$\mathcal{O}_{\mathbb{P}_k^n}(1)$是分次$S$-模
% $S(1) = \bigoplus\limits_{d\geqslant0}S(1)_d$所定义的凝聚层$S(1)^{\sim}$。其中$S(1)$的分次为
% \begin{equation}
% S(1)_d = S_{d+1} = \{ f\in S \ |\ \deg(f) = d+1 \}.
% \end{equation}

下面我们来介绍概型$X$上的 Cartier 除子。

\begin{definition}
设$X$为概型,对$X$的每个仿射开子集$U = \spec(A(U))$,令$K(U)$为$A(U)$的全商环。具体来说,令$S$为$A(U)$中全体非零因子组成的乘法集,那么
\begin{equation}
K(U) = S^{-1}A(U).
\end{equation}
预层$U\mapsto K(U)$实际上是一个层,记作$\mathscr{K}$,被称作$\mathcal{O}_X$的全商环层。此外,用$\mathscr{K}^*, \mathcal{O}_X^*$表示分别环层$\mathscr{K}, \mathcal{O}_X$的可逆元构成的层。
\end{definition}

\begin{definition}
概型$X$上的一个 Cartier 除子指的是层$\mathscr{K}^* / \mathcal{O}_X^*$的一个整体截影。于是概型$X$上所有 Cartier 除子构成了一个群,被记作$\operatorname{Div}(X)$。
\end{definition}
等价地,一个 Cartier 除子$D$可用如下物体来定义:$X$的一个开覆盖$\{U_i\}$,以及$f_i\in K(U_i)^*$,满足$f_i / f_j \in \mathcal{O}_X(U_i\cap U_j)^*$。这些$f_i$被称作 Cartier 除子$D$的局部式。此时,我们也把 Cartier 除子$D$记作$D = \{(U_i, f_i)\}$。

\begin{definition}
Cartier 除子$D = \{(U_i, f_i)\}$被称作有效的,如果它的局部式$f_i\in \mathcal{O}_X(U_i) \subseteq K(U_i)^*$。
\end{definition}

\begin{definition}
概型$X$上的一个 Cartier 除子$D$被称作主除子,如果它在映射$\mathscr{K}^*(X) \to \mathscr{K}^* / \mathcal{O}_X^*(X)$的像中。两个 Cartier 除子被称作是线性等价的,如果他们的差是一个主除子。于是$X$上 Cartier 除子的群$\operatorname{Div}(X)$模掉线性等价关系就得到的商群$\operatorname{CaCl}(X)$被称作 Cartier 除子类群。
\end{definition}

每个 Cartier 除子对应一个$X$上的线丛,从而可以把$X$的 Picard 群与 Cartier 除子类群联系起来。具体来说,有

\begin{definition}
令$D = \{(U_i, f_i)\}$为概型$X$上的一个 Cartier 除子。令$\mathscr{L}(D)$为在$U_i$上由$f_i^{-1}$生成的$\mathscr{K}$的子模,称为与$D$相伴的层。
\end{definition}

\begin{proposition}[\inlinecite{GTM52}, 第II章 Proposition 6.13]
$\mathscr{L}(D)$是$X$上的可逆层。映射$D\mapsto \mathscr{L}(D)$给出了$X$的Cartier 除子群$\operatorname{Div}(X)$与$\mathscr{K}$的可逆子层的群之间的一一对应。
\end{proposition}
如果$D = \{(X, f)\}$是一个主除子,其中$f\in \mathscr{K}^*(X)$,依定义很明显有$\mathscr{L}(D)\cong \mathcal{O}_X$。这个同构由$f^{-1} \leftmapsto 1$给出。于是有单的群同态:
\begin{equation}
\operatorname{CaCl}(X) \longrightarrow \Pic(X): ~~ D \mapsto \mathscr{L}(D).
\end{equation}
特别地,当$X$是域$k$上的射影概型的时候,上述映射是群同构。


% 未完待写!!!$X$到射影空间的映射与除子
概型$X$到射影空间$\mathbb{P}_k^n = \proj k[T_0,\cdots,T_n]$的态射与$X$上的可逆层及其上一组整体截影相互决定,具体地有
\begin{theorem}[\inlinecite{GTM52}, 第II章 Theorem 7.1]
\label{line bundle and morphism into Pn}
\ 
\begin{enumerate}
\item 设$\varphi: X \rightarrow \mathbb{P}_k^n$为概型的态射,那么$\varphi^*\mathcal{O}_{\mathbb{P}_k^n}(1)$是$X$上的可逆层,并且有整体截影$s_i = \varphi^*(T_i)$,$i=0,1,\cdots,n$生成。
\item 若$\mathscr{L}$是$X$上可逆层,并且由一组整体截影$s_0, \cdots, s_n\in \mathscr{L}(X)$生成,那么存在唯一的态射$\varphi: X \rightarrow \mathbb{P}_k^n$,使得$\mathscr{L}\cong \varphi^*\mathcal{O}_{\mathbb{P}_k^n}(1)$,且在此同构下有$s_i = \varphi^*(T_i)$。
\end{enumerate}
\end{theorem}

当$\varphi$是浸入,也就是说$\varphi$给出了$X$的一个拟射影概型结构,那么$\mathscr{L}\cong \varphi^*\mathcal{O}_{\mathbb{P}_k^n}(1)$被称为$X$上的一个极丰沛可逆层。

$\mathcal{O}_{\mathbb{P}_k^n}(1)$为 Serre 扭层。记分次环$S = k[T_0,\cdots,T_n] = \bigoplus\limits_{d\geqslant0}S_d$。那么$\mathcal{O}_{\mathbb{P}_k^n}(1)$是与分次$S$-模$S(1) = \bigoplus\limits_{d\geqslant0}S(1)_d$相伴的凝聚层$S(1)\qcohsheaf$。其中$S(1)$的分次为
\begin{equation}
S(1)_d = S_{d+1} = \{ f\in S \ |\ \deg(f) = d+1 \}.
\end{equation}

对应于概型$X$上的局部自由凝聚层$\mathscr{E}$,有与其相伴的射影空间丛
\begin{equation}
\mathbb{P}(\mathscr{E}) = \PROJ(\sym\mathscr{E}) \overset{\pi}{\longrightarrow} X
\end{equation}
类似之前的定理\ref{line bundle and morphism into Pn},有
\begin{proposition}[\inlinecite{GTM52}, 第II章 Proposition 7.12]
设$g: Y \rightarrow X$为任一态射,则在$X$上给出$Y$到$\mathbb{P}(\mathscr{E})$的一个态射,等价于在$Y$上给出一个可逆层$\mathscr{L}$以及在$Y$上的一个层的满射$g^*\mathscr{E} \rightarrow \mathscr{L}$。
\end{proposition}

\begin{remark}
还可以从更范畴化的语言来看$\mathbb{P}(\mathscr{E})$与$\mathcal{O}_{\mathbb{P}(\mathscr{E})}(1)$。更多的阅读参考文献~\inlinecite{FGA}的第二章。一般地,设$\mathscr{E}$为$X$上的一个向量丛,考虑函子
\begin{equation}
\begin{tikzcd}[row sep = tiny]
Q_{\mathscr{E}}: (Sch/X)^{op} \arrow[r] & (Sets) \\
\phantom{hehehe} (\phi: S \to X) \arrow[r, mapsto] & \{\text{$\phi^*\mathscr{E}$的可逆商丛}\}
\end{tikzcd}
\end{equation}
其中$(Sch/X)^{op}$表示由$X$上所有概型组成的范畴的反范畴,$(Sets)$表示所有集合组成的范畴。$Q_{\mathscr{E}}$是一个可表函子,由泛对象$(\mathbb{P}(\mathscr{E}), \mathcal{O}_{\mathbb{P}(\mathscr{E})}(1))$表示,其中
\begin{equation}
\mathbb{P}(\mathscr{E}) = \PROJ(\sym\mathscr{E}) \overset{\pi}{\longrightarrow} X
\end{equation}
是$X$上的一个射影概型,被称为与向量丛$\mathscr{E}$相伴的射影空间丛。$\mathcal{O}_{\mathbb{P}(\mathscr{E})}(1)$被称作是$\mathbb{P}(\mathscr{E})$上的泛线丛。要注意的是,泛线丛在这里就是 Serre 扭层,而不是它的对偶。
\end{remark}

% \chapter{丢番图几何的一些基础知识}
% \label{apdx: diophantine geometry}

\chapter{Adelic 高度}
\label{apdx: adelic height}
这个附录主要的参考文献是~\inlinecite{LeRudulier2014},~\inlinecite{Peyre}以及~\inlinecite{ZhangShouwu}。设$V$是定义在数域$K$上的一个光滑(等价地,正则)射影簇,$\mathscr{L}$是$V$上的一个线丛。

\begin{definition}[\inlinecite{LeRudulier2014}, D\'{e}finition 2.1] \label{full metric on line bundles}
$V$上的线丛$\mathscr{L}$上的一个范数,指的是一族对象$(\mathscr{L}(x), (\|\cdot\|_v)_{v\in M_K})_{x\in V(\mathbb{C}_v)}$,其中$\mathscr{L}(x) := \mathbb{C}_v \otimes_{\mathcal{O}_V, x}\mathscr{L}_x$为一个$\mathbb{C}_v$-线性空间,$\|\cdot\|_v$是线性空间$\mathscr{L}(x)$上的范数,使得$V$的任一非空开集$U$以及$\mathscr{L}$在$U$上的任一截影$\mathscr{L}(U) \ni s: U \rightarrow \coprod\limits_{x\in U} \mathscr{L}_x$都有
\begin{itemize}
\item 映射$x\mapsto \|s(x)\|_v$在$U(\mathbb{C}_v)$上连续;
\item 对任意的$x\in U(\mathbb{C}_v)$以及任意的$\sigma \in \gal(\mathbb{C}_v / K_v)$有
\begin{equation}
\|\sigma(s)(x)\|_v = \|s(x^{\sigma})\|_v.
\end{equation}
\end{itemize}
$\widetilde{\mathscr{L}} := (\mathscr{L}, (\|\cdot\|_v)_{v\in M_K})$被称为一个赋范线丛。
\end{definition}

\begin{remark}
以上定义中,$s(x)$指的是下面的映射(的像)
\begin{equation}
\begin{tikzcd}
\spec \mathbb{C}_v \arrow[r, "x"] & U \arrow[r, "s"] & \coprod\limits_{x\in U} \mathscr{L}_x \arrow[r, hookrightarrow] & \coprod\limits_{x\in U} \mathbb{C}_v \otimes_{\mathcal{O}_V, x} \mathscr{L}_x
\end{tikzcd}
\end{equation}
而$\sigma(s)$则指的是下面的截影
\begin{equation}
\begin{tikzcd}
U \arrow[r, "s"] & \coprod\limits_{x\in U} \mathscr{L}_x \arrow[r, hookrightarrow] & \coprod\limits_{x\in U} \mathbb{C}_v \otimes_{\mathcal{O}_V, x} \mathscr{L}_x \arrow[r, "\sigma\otimes\identity"] & \coprod\limits_{x\in U} \mathbb{C}_v \otimes_{\mathcal{O}_V, x} \mathscr{L}_x
\end{tikzcd}
\end{equation}
\end{remark}

\begin{remark}
为了方便起见,记号$\|\cdot\|_v$也表示$\gal(\mathbb{C}_v / K_v)$作用下不变的,线丛$\mathscr{L}$到$V$上连续函数层$\mathscr{C}_V$的态射,而且这个态射在每个茎上都是相应线性空间上的满足定义中条件的范数。为了方便起见,$\|\cdot\|_v$有时也还表示以上两个层在$V$的开集上的截面之间的映射。正如定义\ref{full metric on line bundles}中所做的一样。
\end{remark}

\begin{definition}
设$\widetilde{\mathscr{L}} = (\mathscr{L}, (\|\cdot\|_v)_{v\in M_K})$为$V$上的赋范线丛,而且$\mathscr{L}$是$V$上的一个极丰沛线丛,$s_0,\cdots,s_n$是线性空间$\mathscr{L}(V)$的一组基。设$x\in V(\mathbb{C}_v)$,令
\begin{equation}
\delta_v(x) := \log\left( \|s(x)\|_v \max\limits_{0\leqslant i \leqslant n} \left| \dfrac{s_i(x)}{s(x)} \right|_v \right),
\end{equation}
其中$|\cdot|_v$为式\eqref{explicit v-adic abs value}定义的$K_v$上的,从而唯一延拓到$\mathbb{C}_v$上的绝对值,$s\in \mathscr{L}(V)$是$\mathscr{L}$的一个满足$s(x)\neq 0$的整体截影。那么$(\|\cdot\|_v)_{v\in M_K}$被称为 Adelic 范数,如果函数族$(\delta_v(\cdot))_{v\in M_K}$还满足如下条件
\begin{itemize}
\item 对于所有的位点$v\in M_K$,$\delta_v(\cdot)$都是$V(\mathbb{C}_v)$上的有界函数。
\item 除有限个位点外,$\delta_v(\cdot)$在$V(\mathbb{C}_v)$上恒等于$0$。
\end{itemize}
这样的$\widetilde{\mathscr{L}} = (\mathscr{L}, (\|\cdot\|_v)_{v\in M_K})$被称为一个 Adelic 极丰沛线丛。
\end{definition}

% \begin{remark}
% 以上 Adelic 极丰沛线丛的定义中关于函数族$(\delta_v(\cdot))_{v\in M_K}$的要求的第二点,等价地可以写成:除有限个位点外,有
% \begin{equation}

% \end{equation}
% \end{remark}

由于$V$上的一般的一个线丛$\mathscr{L}$总能写成$\mathscr{L} = \mathscr{L}_1\otimes \mathscr{L}_2^{-1}$的形式,其中$\mathscr{L}_1, \mathscr{L}_2$都是$V$上的极丰沛线丛,于是有
\begin{definition}
设$\widetilde{\mathscr{L}}$为$V$上一个赋范线丛,如果它能写成
\begin{equation}
\widetilde{\mathscr{L}} = \widetilde{\mathscr{L}}_1 \otimes \widetilde{\mathscr{L}}_2^{-1},
\end{equation}
其中$\widetilde{\mathscr{L}}_1, \widetilde{\mathscr{L}}_2$都是 Adelic 极丰沛线丛。
\end{definition}

\begin{remark} \label{tensor and dual of adelic line bundle}
$V$上两个赋范线丛$\widetilde{\mathscr{L}}_1 = (\mathscr{L}_1, (\|\cdot\|_{1,v})_{v\in M_K})$,$\widetilde{\mathscr{L}}_2 = (\mathscr{L}_2, (\|\cdot\|_{2,v})_{v\in M_K})$的张量积$\widetilde{\mathscr{L}}_1 \otimes \widetilde{\mathscr{L}}_2$指的是赋范线丛$(\mathscr{L}_1\otimes \mathscr{L}_2, (\|\cdot\|_{1\otimes 2,v})_{v\in M_K})$,其中$\|\cdot\|_{1\otimes 2,v}$是在每个茎上是张量积范数。更精确地说,令$\mathscr{F}$为$V$上预层$U \mapsto \mathscr{L}_1(U)\otimes \mathscr{L}_2(U)$,那么与$\mathscr{F}$相伴的层$\mathscr{F}^+ = \mathscr{L}_1\otimes \mathscr{L}_2$。定义态射$\theta: \mathscr{F} \longrightarrow \mathscr{C}_V$如下:
\begin{equation}
\begin{tikzcd}[row sep = tiny]
\mathscr{F}(U) = \mathscr{L}_1(U)\otimes \mathscr{L}_2(U) \arrow[r, "\theta(U)"] & \mathscr{C}_V(U) \\
\phantom{hehehe} s_1\otimes s_2 \arrow[r, mapsto] & (x \mapsto \|s_1(x)\|_{1,v}\cdot\|s_2(x)\|_{2,v})
\end{tikzcd}
\end{equation}
$\|\cdot\|_{1\otimes 2,v}$便是与上述态射相伴的层的态射:
\begin{equation}
\begin{tikzcd}[row sep = large, column sep = large]
\mathscr{F} \arrow[d] \arrow[r, "\theta"] & \mathscr{C}_V \\
\mathscr{L}_1\otimes \mathscr{L}_2 \arrow[ur, "\|\cdot\|_{1\otimes 2,v}"']
\end{tikzcd}
\end{equation}

$V$上赋范线丛$\widetilde{\mathscr{L}} = (\mathscr{L}, (\|\cdot\|_{v})_{v\in M_K})$的对偶(或者说逆)指的是线丛$\mathscr{L}^{-1} := \mathscr{L}^{\vee} = \sheafhom_{\mathcal{O}_X}(\mathscr{L}, \mathcal{O}_X)$,带上对偶范数,所得的赋范线丛。
\end{remark}

\begin{remark}
Adelic 线丛作为一个良定义的具有良好性质的概念,很自然的要求(或者观察)就是,它与$s$以及$\mathscr{L}(V)$的基$s_0,\cdots,s_n$选取无关,而且应该在射影空间上的向量丛这个范畴上有比较好的性状:
\begin{itemize}
\item 两个 Adelic 线丛的张量积$\widetilde{\mathscr{L}}_1 \otimes \widetilde{\mathscr{L}}_2$(定义见注释\ref{tensor and dual of adelic line bundle})仍然是 Adelic 线丛。
\item Adelic 线丛$\widetilde{\mathscr{L}}$沿光滑射影概型的态射$f: V' \longrightarrow V$的拉回$f^*\widetilde{\mathscr{L}} = {(f^*\mathscr{L}, (f^*\|\cdot\|_{v})_{v\in M_K})}$仍然是$V'$上的 Adelic 线丛。
\end{itemize}
以上两点的详见证明,可以看参考文献~\inlinecite{LeRudulier2014}的 Proposition 2.8 与 Proposition 2.9。
\end{remark}

\begin{definition}
设$\widetilde{\mathscr{L}} = (\mathscr{L}, (\|\cdot\|_v)_{v\in M_K})$为$V$上的 Adelic 线丛。
\begin{enumerate}
\item $V(K)$上相对于赋范线丛$\widetilde{\mathscr{L}}$的 Adelic 高度被定义为
\begin{equation}
H_{\widetilde{\mathscr{L}}, K}(x) = \prod\limits_{v\in M_K} \|s(x)\|_v^{-[K_v:\mathbb{Q}_v]},
\end{equation}
其中$s$是$\mathscr{L}$的截影,使得$s(x)\neq 0$。
\item $V(\overline{\mathbb{Q}})$上的相对于赋范线丛$\widetilde{\mathscr{L}}$绝对 Adelic 高度被定义作
\begin{equation}
H_{\widetilde{\mathscr{L}}}(x) = \prod\limits_{v\in M_K} \|s(x)\|_v^{-[K_v:\mathbb{Q}_v] / [K:\mathbb{Q}]},
\end{equation}
其中$K$是包含点$x$的定义域的一个数域,$s$是$\mathscr{L}$的截影,使得$s(x)\neq 0$。
\end{enumerate}
\end{definition}

\begin{remark}
由乘积公式\eqref{product formula},以上定义的高度与$s$的选取无关。由次数公式\eqref{degree formula},绝对高度的定义与数域$K$选取无关。
\end{remark}

\begin{example}
令$V = \mathbb{P}_K^n = \proj(K[T_0, T_1, \cdots, T_n])$为$n$维射影空间。记$s_0, s_1, \cdots, s_n \in \mathcal{O}_{\mathbb{P}_K^n}(1)(\mathbb{P}_K^n)$为射影坐标$T_0, T_1, \cdots, T_n$所对应的泛丛$\mathcal{O}_{\mathbb{P}_K^n}(1)$的整体截影。设$v\in M_K$为$K$的一个位点。任取$x\in \mathbb{P}_K^n(\mathbb{C}_v)$,以及满足$s(x)\neq 0$的截影$s \in \mathbb{C}_v\otimes \mathcal{O}_{\mathbb{P}_K^n}(1)(\mathbb{P}_K^n)$。令
\begin{equation}
\|s(x)\|_v = \left( \max \left\{ \left| \dfrac{s_0(x)}{s(x)} \right|_v, \left| \dfrac{s_1(x)}{s(x)} \right|_v, \cdots, \left| \dfrac{s_n(x)}{s(x)} \right|_v \right\} \right)^{-1}.
\end{equation}
这样得到的范数是一个 Adelic 范数。特别地,当$\xi = [x_0: x_1: \cdots: x_n] \in \mathbb{P}_K^n(K)$是一个$K$-有理点的时候,假设$x_i\neq 0$,那么取$s = s_i$,有$s_i(\xi) = x_i \neq 0$。于是我们得到一个 Adelic 线丛$\widetilde{\mathcal{O}_{\mathbb{P}_K^n}(1)} = (\mathcal{O}_{\mathbb{P}_K^n}(1), (\|\cdot\|_v)_{v\in M_K})$,对应的高度函数在有理点$\xi$处的值为
\begin{align}
H_{\widetilde{\mathcal{O}_{\mathbb{P}_K^n}(1)}, K}(\xi) & = \prod\limits_{v\in M_K} \|s(x)\|_v^{-[K_v:\mathbb{Q}_v]} = \prod\limits_{v\in M_K} \dfrac{\max\limits_{0\leqslant j \leqslant n} |x_j|_v^{-[K_v:\mathbb{Q}_v]}}{|x_i|_v^{-[K_v:\mathbb{Q}_v]}} \\
& = \prod\limits_{v\in M_K} \max\limits_{0\leqslant j \leqslant n} |x_j|_v^{-[K_v:\mathbb{Q}_v]}.
\end{align}
可以看出,$H_{\widetilde{\mathcal{O}_{\mathbb{P}_K^n}(1)}, K}$限制在$\mathbb{P}_K^n(K)$上得到的就是我们最早定义的经典的高度函数\eqref{relative multiplicative height}是一致的。
\end{example}

关于高度函数$H_{\widetilde{\mathscr{L}}}$的性质可以列举如下;

\begin{proposition}
设$\widetilde{\mathscr{L}} = (\mathscr{L}, (\|\cdot\|_v)_{v\in M_K})$,$\widetilde{\mathscr{L}}' = (\mathscr{L}', (\|\cdot\|_v')_{v\in M_K})$为$V$上的两个 Adelic 线丛,$x\in V(\overline{K})$为$V$上一个代数点,$f: W \longrightarrow V$为光滑射影簇之间的态射。
\begin{enumerate}
\item 如果$\mathscr{L} = \mathscr{L}'$,那么存在正的常数$C_1,C_2$使得
\begin{equation}
C_1H_{\widetilde{\mathscr{L}}'}(x) \leqslant H_{\widetilde{\mathscr{L}}}(x) \leqslant C_2 H_{\widetilde{\mathscr{L}}'}(x).
\end{equation}
\item 对任意的$\sigma \in \gal(\overline{K}/K)$,有
\begin{equation}
H_{\widetilde{\mathscr{L}}}(x^{\sigma}) = H_{\widetilde{\mathscr{L}}}(x).
\end{equation}
\item 对于 Adelic 线丛的张量积对应的高度函数,有
\begin{equation}
H_{\widetilde{\mathscr{L}}\otimes \widetilde{\mathscr{L}}'}(x) = H_{\widetilde{\mathscr{L}}}(x) H_{\widetilde{\mathscr{L}}'}(x).
\end{equation}
\item 对于 Adelic 线丛沿态射的拉回对应的高度函数,有
\begin{equation}
H_{f^*\widetilde{\mathscr{L}}}(x) = H_{\widetilde{\mathscr{L}}}(f(x)).
\end{equation}
\end{enumerate}
\end{proposition}

% \chapter{Arakelov理论的一些基础知识}
% \label{apdx: arakelov theory}
% 这一章附录中,我们介绍一些 Arakelov 理论的基础知识,用于定义射影空间中点的Arakelov高度。主要的参考资料是陈华一教授于2010年在清华Yau数学中心的暑期学校讲授的课程``Arakelov 几何''的课程讲义,以及参考文献~\inlinecite{Moriwaki-book}。

% 在这一章中,我们取定$一个数域K$,记其整数环为$\mathcal{O}_K$,并记$S = \spec\mathcal{O}_K$。

% \section{数域上的 Hermite 向量丛}
% \label{apdx: Hermite vector bundle over number field}
% \begin{definition} \label{Hermite vector bundle}
% $S$上的赋范向量丛指的是一对物体$\overline{E} = (E,h)$,其中
% \begin{itemize}
% \item $E$是秩为$r$的局部自由(或者说,投射)$\mathcal{O}_K$-模,$r\in \mathbb{N}^+$。$r$也被称为赋范向量丛$\overline{E}$的秩。
% \item $h = (\|\cdot\|_{v})_{v\in M_{K, \infty}}$是一族范数,其中$\|\cdot\|_{v} = \|\cdot\|_{E,v}$是复线性空间$E\otimes_{\mathcal{O}_K, v} \mathbb{C}$上的范数,且在 Galois 群$\gal(\mathbb{C}/K_v)$作用下不变。
% \end{itemize}
% 如果$h$中所有范数都是 Hermite 范数,则称赋范向量丛$\overline{E}$为一个 Hermite 向量丛。秩为$1$的赋范向量丛自然是 Hermite 向量丛,被称为 Hermite 线丛。
% \end{definition}

% 要注意这里赋范向量丛的含义与上一章附录\ref{apdx: adelic height}中的所定义的赋范线丛的含义的异同。

% % Hermite 范数可以通过标准的方式唯一对应一个 Hermite 内积,于是赋范向量丛$\overline{E}$也对应了一族 Hermite 内积$(\langle\cdot,\cdot\rangle_{E,v})_{v\in M_{K, \infty}}$。

% \begin{definition}
% 设$\overline{E} = (E, (\|\cdot\|_{E,v})_{v\in M_{K, \infty}})$为$S$上的赋范向量丛。
% \begin{enumerate}
% \item $\overline{E}$的一个子丛,指的是$E$的一个饱和的子模$F$,以及范数$(\|\cdot\|_{F,v})_{v\in M_{K, \infty}}$组成的赋范向量丛,其中$\|\cdot\|_{F,v}$是$\|\cdot\|_{E,v}$在复线性空间$E \otimes_{\mathcal{O}_K, v} \mathbb{C}$的子空间$F \otimes_{\mathcal{O}_K, v} \mathbb{C}$上的限制。
% \item $\overline{E}$的一个商丛,指的是$E$的一个投射商模$G$以及范数$(\|\cdot\|_{G,v})_{v\in M_{K, \infty}}$组成的赋范向量丛,其中$\|\cdot\|_{G,v}$是$\|\cdot\|_{E,v}$在复线性空间$E \otimes_{\mathcal{O}_K, v} \mathbb{C}$的商空间$G \otimes_{\mathcal{O}_K, v} \mathbb{C}$上诱导的商范数。
% \item $\overline{E}$的对偶丛$\overline{E}^{\vee}$,是由$E$的对偶模$E^{\vee} = \Hom_{\mathcal{O}_K}(E, \mathcal{O}_K)$,以及对偶范数组成的赋范向量丛:
% \begin{equation} \label{dual norm}
% \|f\|_v := \sup\limits_{x\neq0}\dfrac{\|f(x)\|_v}{\|x\|_v}.
% \end{equation}
% \item 赋范向量丛的直和$\overline{E}_1 \oplus \overline{E}_2$指的是由$\mathcal{O}_K$-模$E_1\oplus E_2$以及复线性空间直和$(E_1 \otimes_{\mathcal{O}_K, v} \mathbb{C}) \oplus (E_2 \otimes_{\mathcal{O}_K, v} \mathbb{C}) \cong (E_1\oplus E_2)\otimes_{\mathcal{O}_K, v} \mathbb{C})$上的范数
% \begin{equation}
% \|\cdot\|_{E_1\oplus E_2, v}: (x,y) \mapsto \sqrt{\|x\|_{E_1,v}^2+\|y\|_{E_2,v}^2}
% \end{equation}
% 给出的赋范向量丛。
% \end{enumerate}
% \end{definition}
% 若赋范向量丛$\overline{E}$是 Hermite 向量丛,那么它的子丛,商丛,对偶,直和都是 Hermite 向量丛。

% Hermite 范数可以通过标准的方式唯一对应一个 HermiteHermite 内积,于是赋范向量丛$\overline{E}$也对应了一族 Hermite 内积$(\langle\cdot,\cdot\rangle_{E,v})_{v\in M_{K, \infty}}$。于是 Hermite 向量丛的张量积$\overline{E}_1 \otimes \overline{E}_2$可以被定义为$\mathcal{O}_K$-模$E_1 \otimes E_2$带上相应复线性空间上如下定义的张量积范数
% \begin{equation} \label{tensor norm}
% \langle x_1\otimes y_1, x_2\otimes y_2 \rangle = \langle x_1, x_2 \rangle\langle y_1, y_2 \rangle.
% \end{equation}
% Hermite 向量丛的外积$\wedge^r\overline{E}$可以被定义为$\mathcal{O}_K$-模$\wedge^r E$加上相应复线性空间上如下定义的外积范数
% \begin{equation} \label{wedge norm}
% \langle x_1\wedge\cdots\wedge x_r, y_1\wedge\cdots\wedge y_r \rangle = \det(\langle x_i, y_j \rangle)_{1\leqslant i,j \leqslant r}.
% \end{equation}

% \section{Arakelov 度数}
% \label{apdx: arakelov degree}
% \begin{definition}
% 设$\overline{L} = (L, (\|\cdot\|_{v})_{v\in M_{K, \infty}})$为$S = \spec\mathcal{O}_K$上的 Hermite 线丛,$s$是$L\otimes_{\mathcal{O}_K}K$中的任意一个非零元。定义$\overline{L}$的 Arakelov 度数为
% \begin{align} \label{arakelov degree}
% \widehat{\deg}(\overline{L}) & := -\sum\limits_{v\in M_K} [K_v:\mathbb{Q}_v]\log(\|s\|_v) \\
% & = \log(\#(L/s\mathcal{O}_K)) - \sum\limits_{v\in M_{K, \infty}} [K_v:\mathbb{Q}_v]\log(\|s\|_v)
% \end{align}
% \end{definition}

% \begin{remark}\ 
% \begin{enumerate}
% \item 由于$\overline{L}$秩为$1$,对任意一个$s' \in (L\otimes_{\mathcal{O}_K}K)^*$,存在域$K$中非零元$a$,使得$s' = as$,于是根据乘积公式\eqref{product formula}容易看出 Hermite 线丛$\overline{L}$的 Arakelov 度数的定义式\eqref{arakelov degree}与$s$的选取无关。
% \item 由范数的性质容易推出$\widehat{\deg}(\overline{L}_1 \otimes \overline{L}_2) = \widehat{\deg}(\overline{L}_1) + \widehat{\deg}(\overline{L}_2)$对任意两个 Hermite 线丛$\overline{L}_1, \overline{L}_2$成立。同样地,又对偶范数的定义\eqref{dual norm}以及乘积公式\eqref{product formula},可以推出$\widehat{\deg}(\overline{L}^{\vee}) = -\widehat{\deg}(\overline{L})$。
% \end{enumerate}
% \end{remark}

% \begin{definition}
% 设$\overline{E} = (E, (\|\cdot\|_{v})_{v\in M_{K, \infty}})$为秩为$r$的 Hermite 向量丛。$\overline{E}$的行列式丛被定义为$X$上 Hermite 线丛
% \begin{equation}
% \det(\overline{E}) := \wedge^r \overline{E}.
% \end{equation}
% $\overline{E}$的 Arakelov 度数被定义为
% \begin{equation}
% \widehat{\deg}(\overline{E}) := \widehat{\deg}(\det(\overline{E})).
% \end{equation}
% \end{definition}
% 具体来说,设$s_1,\cdots,s_r\in E$为线性空间$E\otimes_{\mathcal{O}_K} K$的一组基,那么
% \begin{align}
% \widehat{\deg}(\overline{E}) & = -\sum\limits_{v\in M_K} [K_v:\mathbb{Q}_v]\log(\|s_1\wedge\cdots\wedge s_r\|_v) \\
% & = \log(\#(E/(\sum\limits_{i=1}^r s_i\mathcal{O}_K))) - \dfrac{1}{2}\sum\limits_{v\in M_{K, \infty}} [K_v:\mathbb{Q}_v]\log(\det(\langle s_i, s_j \rangle)_{1\leqslant i,j \leqslant r}).
% \end{align}

% \begin{definition}
% $S = \spec\mathcal{O}_K$上的 Hermite 向量丛$\overline{E}$的规范化的 Arakelov 度数被定义为
% \begin{equation}
% \widehat{\deg}_n(\overline{E}) := \dfrac{1}{[K:\mathbb{Q}]} \widehat{\deg}(\overline{E}).
% \end{equation}
% \end{definition}
% 规范化的 Arakelov 度数在域扩张下不变:
% \begin{proposition} \label{invariance of normalized arakelov height}
% 设$L/K$为数域的扩张,$\pi: \spec\mathcal{O}_L \to \mathcal{O}_K$为自然态射,$\overline{E}$为$\spec\mathcal{O}_K$上的 Hermite 向量丛,那么
% \begin{equation}
% \widehat{\deg}_n(\pi^*\overline{E}) = \widehat{\deg}_n(\overline{E}).
% \end{equation}
% \end{proposition}
% $\pi^*\overline{E}$表示$\overline{E}$沿$\pi$的拉回,是由以下物体给出的$\spec\mathcal{O}_L$上的 Hermite 向量丛:其$\mathcal{O}_L$-模部分是$E\otimes_{\mathcal{O}_K}\mathcal{O}_L$;对于$v\in M_{K,\infty}, w\in M_{L,\infty}, w|v$,有$\pi^*E\otimes_{\mathcal{O}_L, w} \mathbb{C} \cong E\otimes_{\mathcal{O}_K, v} \mathbb{C}$,其范数$\|\cdot\|_{\pi^*E,w}$取作$\|\cdot\|_{E,v}$。

% \section{射影概型上的 Hermite 向量丛}
% \label{apdx: Hermite vector bundle over projective scheme}
% 设$\pi: \mathscr{X} \to S$是$S = \spec\mathcal{O}_K$上的平坦射影概型,也就是说$\pi$有分解
% \begin{equation}
% \begin{tikzcd}
% \proj(\mathcal{O}_K[T_0,\cdots,T_n]/I) \cong \mathscr{X} \arrow[rr, "\pi"] \arrow[dr, hookrightarrow] & & \spec\mathcal{O}_K \\
% & \proj(\mathcal{O}_K[T_0,\cdots,T_n]) \arrow[ur]
% \end{tikzcd}
% \end{equation}
% 其中$n\in\mathbb{N}^+$,$I$为$\mathcal{O}_K[T_0,\cdots,T_n]$的一个齐次理想。
% \begin{definition}
% $\mathscr{X}$上的 Hermite 线丛$\overline{\mathscr{L}}$由一对数学实体$(\mathscr{L}, (\|\cdot\|_{v})_{v\in M_{K,\infty}})$组成,其一是一个可逆$\mathcal{O}_{\mathscr{X}}$-模$\mathscr{L}$,另一个是一族度量$(\|\cdot\|_{v})_{v\in M_{K,\infty}}$,使得每一个$\|\cdot\|_{v}$都是复解析空间$\mathscr{X}_v^{an}(\mathbb{C}) := (\mathscr{X}\times_{S,v}\spec\mathbb{C})^{\gaga}$上的线丛$\mathscr{L}_v^{an}(\mathbb{C})$上的$\gal(\mathbb{C}/K_v)$-不变连续度量。
% \end{definition}

% \begin{remark}
% 以上定义中,记$\pr_1: \mathscr{X}_v^{an}(\mathbb{C}) = \mathscr{X}\times_{S,v}\spec\mathbb{C} \to \mathscr{X}$为第一分量的投影态射,$\pr_1^*$为模层的拉回函子,那么$\mathscr{L}_v^{an}(\mathbb{C}) = (\pr_1^*(\mathscr{L}))^{\gaga}$。而$\|\cdot\|_{v}$则是$\mathscr{L}_v^{an}(\mathbb{C})$到$\mathscr{X}_v^{an}(\mathbb{C})$上的连续实值函数层$\mathscr{C}_{\mathscr{X}_v^{an}(\mathbb{C})}$的态射,在每一点$z\in \mathscr{X}_v^{an}(\mathbb{C})$的茎上看,$\|\cdot\|_{v,z}: \mathscr{L}_z \to \mathbb{R}$是复线性空间的范数。
% \end{remark}

% \begin{remark} \label{pullback of hermitian line bundle}
% 设$x\in \mathscr{X}(K)$是一个点。由于$\pi: \mathscr{X} \to \spec\mathcal{O}_K$是平坦射影的,$x$可以唯一地延拓成$\mathscr{X}(\mathcal{O}_K)$的点$\mathcal{P}_x$:
% \begin{equation}
% \begin{tikzcd}
% \mathscr{X}_v^{an}(\mathbb{C}) \arrow[drr, bend left = 20] \arrow[dd, bend right = 20] & & & \\
% & \mathscr{X}_K \arrow[r, hookrightarrow, "i"] & \mathscr{X} \arrow[r, "\pi"] & \spec\mathcal{O}_K \arrow[dl, equal] \\
% \spec\mathbb{C} \arrow[r, "v"] & \spec K \arrow[u, "x"] \arrow[r] & \spec\mathcal{O}_K \arrow[u, dashed, "\mathcal{P}_x"']
% \end{tikzcd}
% \end{equation}
% 可逆$\mathcal{O}_{\mathscr{X}}$-模$\mathscr{L}$沿$\mathcal{P}_x$的拉回$\mathcal{P}_x^*\mathscr{L}$(的整体截影)便是一个局部自由秩为$1$的$\mathcal{O}_K$-模,而对于每个$v\in M_{K, \infty}$,$\mathscr{X}_v^{an}(\mathbb{C})$上的线丛$\mathscr{L}_v^{an}(\mathbb{C})$前推到$\spec\mathbb{C}$上,$\|\cdot\|_v$会诱导出一维复线性空间$\mathcal{P}_x^*\mathscr{L} \otimes_{\mathcal{O}_K, v} \mathbb{C}$上的 Hermite 范数。这样,我们得到了$\spec \mathcal{O}_K$上的一个 Hermite 线丛,记作$\mathcal{P}_x^* \overline{\mathscr{L}}$。
% \end{remark}

% \begin{remark}
% 和数域上的 Hermite 线丛类似,可以定义$\mathscr{X}$上的 Hermite 线丛$\overline{\mathscr{L}}$的对偶线丛$\overline{\mathscr{L}}^{\vee}$,以及两个 Hermite 线丛$\overline{\mathscr{L}}_1$与$\overline{\mathscr{L}}_2$的张量积$\overline{\mathscr{L}}_1\otimes\overline{\mathscr{L}}_2$。在张量积运算下,$\mathscr{X}$上所有的 Hermite 线丛的同构类构成了一个 Abel 群,记作$\widehat{\Pic}(\mathscr{X})$。$\overline{\mathscr{L}}$在这个群里的逆元就是$\overline{\mathscr{L}}^{\vee}$。
% \end{remark}


% % $\mathcal{P}_x$同时也可以被视为一个态射$\mathcal{P}_x: \spec\mathcal{O}_K \to \mathscr{X}$。可逆$\mathcal{O}_{\mathscr{X}}$-模$\mathscr{L}$沿$\mathcal{P}_x$的拉回$\mathcal{P}_x^*\mathscr{L}$便是一个局部自由秩为$1$的$\mathcal{O}_K$-模,再加上由$\|\cdot\|_v$
% % 更确切地说,$\mathcal{P}_x$是态射$x: \spec K \to \mathscr{X}$与$\spec K \to \spec \mathcal{O}_K$的推出,或者说下图是 cocartesian 的
% % $$
% % \begin{tikzcd}
% % \mathscr{X} \bigsqcup_{\spec K} \spec \mathcal{O}_K & \spec \mathcal{O}_K \arrow[l, "\mathcal{P}_x"'] \\
% % \mathscr{X} \arrow[u] & \spec K \arrow[u] \arrow[l, "x"]
% % \end{tikzcd}


\chapter{相交理论的一些基础知识}
\label{apdx: intersection theory}

本章附录主要介绍第\ref{chapter:geometric}章需要用到的相交理论的一些基础的知识。参考文献~\inlinecite{GTM52}的附录A就可以覆盖这部分的内容。更一般的讨论可见~\inlinecite{Fulton}。

设$X$为域$k$上的代数簇。

\begin{definition}
代数簇$X$上的余维$r$的环元指的是,$X$的余维数等于$r$的所有闭的不可约子簇自由生成的 Abel 群中的一个元素,一个有限的形式和
\begin{equation}
\alpha = \sum n_i V_i.
\end{equation}
$X$上的所有余维$r$的环元组成的集合被记为$Z^r(X)$。
\end{definition}

\begin{example}
假设$W$是$X$的任一闭子概型,且有不可约分支$W_1,\cdots,W_t$,其广点分别为$\eta_1,\cdots,\eta_t$,那么$W$的基本环元被定义为
\begin{equation}
[W] = \sum\limits m_i W_i,
\end{equation}
其中$m_i = \ell_{\mathcal{O}_{W_i,\eta_i}}(\mathcal{O}_{W_i,\eta_i})$被称为$W_i$的几何重数。
\end{example}

\begin{example}
设$W$为$X$的余维数等于$r-1$的子簇,其广点为$\eta$,那么
\begin{equation}
K(W) := \mathcal{O}_{W,\eta}
\end{equation}
是一个域,被称作$W$的函数域。任取$W$的一个在$W$中余维数等于$1$的子簇$V$,设其广点为$\eta_V$。令$A = \mathcal{O}_{W,\eta_V}$,为一个整环。那么$K(W) = \operatorname{Frac}(A)$为$A$的分式域。任取$f\in K(W)^*$,则$f$就可以表示为$f = \dfrac{a}{b}$的形式,其中$a,b\in A$。定义整数
\begin{equation} \label{ord_V(f)}
\ord_V(f) = \ord_V(\dfrac{a}{b}) = \ord_V(a) - \ord_V(b) = \ell_{A}(A/aA) - \ell_{A}(A/bA),
\end{equation}
其中$\ell_A(\cdot)$为\ref{length of a module}中定义的$A$-模的长度。可以证明$\ord_V(\cdot)$是一个良定义的群同态
\begin{equation}
\ord_V: K(W) \longrightarrow \mathbb{Z}
\end{equation}
于是我们可以定义一个与$f$相伴的余维$r$的环元
\begin{equation}
[\operatorname{div}(f)] = \sum \ord_V(f) V,
\end{equation}
求和号下的$V$遍历$W$所有的余维数等于$1$的子簇。
\end{example}

\begin{definition}
称$\alpha \in Z^r(X)$为有理等价于$0$,记作$\alpha \sim 0$,如果$\alpha$可表示为有限和
\begin{equation}
\alpha = \sum [\operatorname{div}(f_i)],
\end{equation}
其中$f_i\in K(W_i)^*$,$W_i$为$X$的余维数等于$r-1$的子簇。
\end{definition}

把群$Z^r(X)$中所有有理等价于$0$的元素组成的集合记作$\operatorname{Rat}^r(X)$。由定义式\eqref{ord_V(f)}很容易知道$[\operatorname{div}(f^{1})] = -[\operatorname{div}(f)]$,于是$\operatorname{Rat}^r(X)$是$Z^r(X)$的一个子群。

\begin{definition}
称商群
\begin{equation}
\operatorname{CH}^r(X) = Z^r(X) / \operatorname{Rat}^r(X)
\end{equation}
为$X$上余维$r$的环元的有理等价类群。定义正则代数簇$X$的 Chow 群为分次群
\begin{equation}
\operatorname{CH}(X) = \bigoplus\limits_{r=0}^{\dim(X)} \operatorname{CH}^r(X).
\end{equation}
\end{definition}

\begin{remark}
对于正则射影概型$X$,有
\begin{gather}
\operatorname{CH}^0(X) \cong \mathbb{Z} \\
\operatorname{CH}^1(X) \cong \operatorname{CaCl}(X) \cong \Pic(X)
\end{gather}
设$\mathscr{L}$为$X$上的线丛,$V$是任意一个余维数等于$r$的子簇。那么$\mathscr{L}|_V \cong \mathscr{L}(D)$,其中$D$是$V$上的一个 Cartier 除子,在不计线性等价的意义下唯一,从而对应于$\operatorname{CH}^{r+1}(X)$的一个元素,记作$[D]$。令(见~\inlinecite{Fulton} \S 2.5)
\begin{equation}
c_1(\mathscr{L})\cap V = [D].
\end{equation}
于是通过自然的延拓,有良定义的映射
\begin{equation}
c_1(\mathscr{L})\cap -: \operatorname{CH}^{r}(X) \longrightarrow \operatorname{CH}^{r+1}(X).
\end{equation}
通过映射的复合与相加,对于任意一个$d$次齐次的多项式$P(T_1,\cdots,T_n)$,以及$X$上的线丛$\mathscr{L}_1,\cdots,\mathscr{L}_n$,$n\leqslant \dim(X)-r$,有映射
\begin{equation}
P(c_1(\mathscr{L}_1),\cdots,c_1(\mathscr{L}_n)) \cap -: \operatorname{CH}^{r}(X) \longrightarrow \operatorname{CH}^{r+n}(X)
\end{equation}
\end{remark}

\begin{definition}[\inlinecite{Fulton}, Definition 1.4]
设$X$为域$k$上的射影簇,$\alpha = \sum n_PP\in Z^{\dim(X)}(X)$,那么$\alpha$的次数被定义为
\begin{equation}
\deg(\alpha) = \int\limits_{X} \alpha = \sum n_P[K(P):K].
\end{equation}
\end{definition}

\begin{remark}
$\deg(\cdot)$可以导出为$\operatorname{CH}^{\dim(X)}(X)$上的映射。
\end{remark}

\begin{remark}
由于$c_1(\mathscr{L})^{\dim(X)} \cap X \in \operatorname{CH}^{\dim(X)}(X)$,于是我们可以定义次数
\begin{equation}
\deg_{\mathscr{L}}(X) = \deg(c_1(\mathscr{L})^{\dim(X)} \cap X).
\end{equation}
对于$X$是$\mathbb{P}_k^n$的纯维数的子概形,取$\mathscr{L} = \mathcal{O}_1$的时候,以上定义的次数与通过 Hilbert 多项式得到的次数是一致的。
\end{remark}

\begin{remark}
正则射影簇$X$的 Chow 群$\operatorname{CH}(X)$在第\ref{chapter:geometric}章中定义的相交积下成为一个结合的含幺交换环,称为$X$的 Chow 环。
\end{remark}


% \begin{proposition}[\inlinecite{GTM52} \S A.2]
% 设$\mathscr{E}$为正则射影簇$X$上秩等于$r$的局部自由向量丛,$\mathbb{P}(\mathscr{E}) \xrightarrow{\pi} X$为$\mathscr{E}$相伴的射影空间丛,$\mathcal{O}_{\mathbb{P}(\mathscr{E})}(1)$为其上的泛线丛(详见本文附录\ref{apdx: algebraic geometry})。令$\alpha\in \operatorname{CH}^1(\mathbb{P}(\mathscr{E})) \cong \Pic(\mathbb{P}(\mathscr{E}))$对应于$\mathcal{O}_{\mathbb{P}(\mathscr{E})}(1)$所在的类。那么$\pi^*$使$\operatorname{CH}(\mathbb{P}(\mathscr{E}))$成为由$1, \alpha, \alpha^2, \cdots, \alpha^r$生成的自由$\operatorname{CH}(X)$-模。
% \end{proposition}

% \begin{definition}
% 设$\mathscr{E}$为正则射影簇$X$上秩等于$r$的局部自由向量丛,对于$i = 0, 1, \cdots, r$,归纳地定义第$i$个 Chern 类$c_i(\mathscr{E}) \in \operatorname{CH}^i(X)$如下:
% \begin{itemize}
% \item $c_0(\mathscr{E}) = 1$,
% \item $c_0(\mathscr{E}) = 1$
% \end{itemize}
% \end{definition}






